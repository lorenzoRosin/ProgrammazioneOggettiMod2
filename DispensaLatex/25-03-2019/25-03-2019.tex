

\newpage
\section{25-03-2019: LIST E SET}
\noindent Ha continuato quello che ha fatto la scorsa lezione... in più ha spiegato le interfacce \textit{comparable} e \textit{comparator} \newline 
\textbullet\ \textit{Comparable}: ha un unico metodo compareTo(T obj) che compara se stesso con un oggetto di tipo T \newline
\textbullet\ \textit{Comparator}: ha molti metodi, il metodo compare(T a, T b) ci permette di realizzare una classe che compara oggetti tra di loro. \newline
Vediamo un esempio della classe Sort: \newline
\begin{lstlisting}[basicstyle=\small,]
Sort(T a) /* Posso usare questo metodo quando 
			 gli oggetti sono già comparabili */
Sort(T a, Comparator<T>) /* Uso questo metodo 
		     quando voglio rendere gli oggetti
		     comparabili al momento */
\end{lstlisting}

Inserisco il codice caricato dal professore quel giorno, esso o modifica classi già definite la lezione precedente o ne aggiunge di nuove: 

\begin{lstlisting}[basicstyle=\small,]
/* Classe: MySortedSet.java */
public interface MySortedSet<T extends Comparable<T>> extends MySet<T> {

}
\end{lstlisting}

\begin{lstlisting}[basicstyle=\small,]
/* Classe: MyArrayListSortedSet.java */
import java.util.*;
import java.util.function.Function;

public class MyArrayListSortedSet<T extends Comparable<T>>
        extends MyAbstractArrayListSet<T> {

    public MyArrayListSortedSet() {
        super();
    }

    @Override
    protected void sort() {
        Collections.sort(a);
    }

}
\end{lstlisting}

\begin{lstlisting}[basicstyle=\small,]
/* Classe: MyArrayListSet.java */
/* è stata modificata la classe dell'altra volta*/

import java.util.*;
import java.util.function.Function;

public class MyArrayListSet<T> extends MyAbstractArrayListSet<T> {

    private final Comparator<T> p;

    public MyArrayListSet(Comparator<T> p) {
        super();
        this.p = p;
    }

    @Override
    protected void sort() {
        Collections.sort(a, p);
    }

}

\end{lstlisting}

\begin{lstlisting}[basicstyle=\small,]
/* Classe: MyAbstractArrayListSet.java */
import java.util.*;
import java.util.function.Function;

public abstract class MyAbstractArrayListSet<T> implements MySet<T> {
    protected final ArrayList<T> a;

    protected MyAbstractArrayListSet() {
        this.a = new ArrayList<T>();
    }

    @Override
    public void add(T x) {
        if (!contains(x)) {
            a.add(x);
            sort();
        }
    }

    protected abstract void sort();

    @Override
    public void clear() {
        a.clear();
    }

    @Override
    public void remove(T x) {
        a.remove(x);
    }

    @Override
    public boolean contains(T x) {
        return a.contains(x);
    }

    @Override
    public boolean contains(Function<T, Boolean> p) {
        return a.contains(p);
    }

    @Override
    public int size() {
        return a.size();
    }

    @Override
    public MyIterator<T> iterator() {
        Iterator<T> it = a.iterator();
        return new MyIterator<T>() {

            @Override
            public boolean hasNext() {
                return it.hasNext();
            }

            @Override
            public T next() {
                return it.next();
            }
        };
    }

    @Override
    public int find(T x) throws Exception {
        int r = a.indexOf(x);
        if (r < 0) throw new Exception("not found");
        return r;
    }
}

\end{lstlisting}

\begin{lstlisting}[basicstyle=\small,]
/* Classe: Main.java */
import java.util.*;

public class Main {

    public static class Plant {
        private int height;

    }

    public static class Animal implements Comparable<Animal> {
        private int weight;
        private String name;

        public Animal(String name, int w) {
            this.name = name;
            this.weight = w;
        }

        public int getWeight() { return weight; }

        public String getName() {
            return name;
        }

        @Override
        public int compareTo(Animal o) {
            return this.weight - o.weight;
            /*if (this.weight == o.weight) return 0;
            else if (this.weight > o.weight) return 1;
            else return -1;*/
        }
    }

    public static class Dog extends Animal {

        public Dog(String name, int w) {
            super(name, w);
        }
    }

    public static void main(String[] args) {

        List<Animal> a = new ArrayList<>();
        Collections.sort(a);

        List<Plant> b = new ArrayList<>();
        Collections.sort(b, new Comparator<Plant>() {
            @Override
            public int compare(Plant x, Plant y) {
                return x.height - y.height;
            }
        });

    }
}
\end{lstlisting}