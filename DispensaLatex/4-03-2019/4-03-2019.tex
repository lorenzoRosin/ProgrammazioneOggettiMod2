

\newpage
\section{4-03-2019: CLASSI ANONIME, LAMBDA E FUNZIONI ORDINE SUPERIORE}
\textbf{DESIGN PATTERN} \newline
\textbullet\ Iteratore \newline
\textbullet\ Compact o Callback o Unary function 

\noindent \textbf{CLASSI ANONIME} \newline
Le classi \textit{lambda} servono per fare funzioni al "volo",senza quindi avere il bisogno di implementare delle interfacce in classi separate. Vediamone un esempio:

\begin{lstlisting}[basicstyle=\small,]
interface HelloWorld {
	public void greet();
    public void greetSomeone(String someone);
}

public static void main (String args[]) {
    HelloWorld frenchGreeting = new HelloWorld() {
        String name = "tout le monde";
        public void greet() {
            greetSomeone("tout le monde");
        }
        public void greetSomeone(String someone) {
            name = someone;
            System.out.println("Salut " + name);
        }
    };
}

\end{lstlisting}

\noindent \textbf{LAMBDA ASTRAZIONI} \newline
Le espressioni \textit{lambda} servono per fare funzioni al "volo",senza quindi avere il bisogno di implementare delle interfacce in classi separate, classi anonime e classi innestate. Si ricorda che le espressioni lambda possono essere usate solo per implementare interfacce funzionali (cioè interfacce che possiedono un solo metodo astratto). Vediamone un esempio:

\begin{lstlisting}[basicstyle=\small,]
interface MyString {
	String myStringFunction(String str);
}

public static void main (String args[]) {
	MyString reverseStr = (str) -> {
		String result = "";		
		return result;
	};
	reverseStr.myStringFunction("temp");
}
\end{lstlisting}

\noindent \textbf{FUNZIONI DI ORDINE SUPERIORE} \newline
Sono delle funzioni che prendono delle funzioni come parametri di ingresso 
\begin{lstlisting}[basicstyle=\small,]
/* questa interfaccia è equivalente all'interfaccia java.util.Functional
 * essa infatti espone una serie di funzioni, senza implementazione, tra 
 * cui anche le call-back */
public interface Func<A, B>{
	B execute(A a); 
	
	/* questa è l'unica funzione esposta dall'interfaccia
	 * un altro nome ragionevole per il metodo execute() è apply() 
	 * oppure call() il nome deve ricordare il fatto di richiamare 
	 * la funzione  */
}

/* questa funzione va ad utilizzare la funzione Func definita sopra */
public static <A,B> List<B> map(List<A> I, Func<A,B> f){
	List<B> r = new ArrayList<>();
	for(A x: l)
		r.add(f.execute(x));
	return r;

}
\end{lstlisting}
A e B sono \textit{generics} locali al metodo(e solo al metodo) \newline
I generics sulle classi servono per parametrizzare, non per fare polimorfismo \newline
\begin{lstlisting}[basicstyle=\small,]
public static <A,B> List<B> map(List<A> I, Func<A,B> f){
/* dove in "public static <A,B>" dichiaro i parametri che useremo
 * mentre in "List<a> .. Func <A,B>" "uso" i parametri */
\end{lstlisting}

\noindent Funzione FILTER: 
\begin{lstlisting}[basicstyle=\small,]
/* questa funzione è simile alla funzione Func definita sopra
 * solo che rende più chiaro il suo scopo: filtrare degli elementi
 * da aggiungere in una lista */
public static <A> List<A> Filter (List<A> l, Func<A,Boolean> p){
	List<A> r = new ArrayList<>();
	for(A x : l)
		if(p.execute(x))
			r.add(x);
	return r;
}
\end{lstlisting}

\noindent La seguente \textit{Filter2} NON funziona perché usa la \textit{remove()} delle Collection, ma non è possibile rimuove un elemento in fase di scorrimento (è scritto nella documentazione)

\begin{lstlisting}[basicstyle=\small,]
public static <A> void Filter2 (List<A>, Func<A, Boolean> p){
	for(A a: l)   /* il for each in Java essere zucchero sintattico */
		if(p.execute(a))
			l.remove(a);

}
\end{lstlisting}
Se non posso rimuovere come ho fatto sopra un elemento posso invece chiedere all'iteratore di rimuovere l'elemento stesso, esso rimuoverà quello a cui stiamo puntando. Quindi invece di usare un ciclo for come sopra, uso l'iteratore per scorrere la lista usando in metodo \textit{hasNext}.

\begin{lstlisting}[basicstyle=\small,]
/* questo funziona perchè chiama la remove() dell'iteratore
 * mentre prima chiamavo il remove della lista */
public static <A> void Filter2 (List<A>l, Func<A, Boolean> p){
	Iterator<A> it = l.iterator();
	while(it.hasNext()){
		A a = it.next();
		if(!(p.execute(a)))
			it.remove();
	}
}

/* volendo posso usare le funzionalita delle nuove API FUNZIONALI
 * l.removeIf(a -> !p.execute(a)); */
\end{lstlisting}

\noindent Posso usare Function<A,B> di java come funziona Func? \newline
Esempio di chiamata:

\begin{lstlisting}[basicstyle=\small,]
public static void main(String argv[]){
	List<String> strings = new ArrayList<>();
	string.add("ciao");
	string.add("pippo");
	string.add("unive");
	/* voglio calcolare la lunghezza della mia lista */
	List<Integer> r = map(strings, new Func<String, Integers>{ 
	
		/* "String" è la lista, "Integers" è la funzione
         * In realtà devo passare un oggetto di tipo Func<>
         * alla funzione map, perciò passo una classe anonima,
         * all'interno della quale trovo solo un metodo
         * (che ha la stessa firma del metodo dell'interfaccia Func)*/
         
		@Override 
		public Integer execute(String a){
			return a.length();
		}	
	});
}
\end{lstlisting}


\noindent La seguente funzione data una lista di interi scarta gli elementi minori di zero: questo è il modo per non usare un for con un ciclo if innestato.

\begin{lstlisting}[basicstyle=\small,]
public static void main__filter(){
	 List<Integer> interi  = new ArrayList<>();
	 interi.add(89);
	 interi.add(34);
	 interi.add(-16);
	 interi.add(560);
	 interi.add(-1);
	 interi.add(46);
	 /* filter prende una lista e un predicato e produce una lista in uscita */
	 List<Integer> l = Filter(ints, new Func<Integer, Boolean>(){
	 	@Override
	 	public Boolean execute (Integer a){
	 		return a>=0;
	 	}
	 });
}
\end{lstlisting}

\noindent Oppure 

\begin{lstlisting}[basicstyle=\small,]
	Filter2 (interi , new Func<Integer, Boolean>(){
		@Override
		public Boolean execute (Integer a)
			return a>=0;
	})
\end{lstlisting}

\noindent \textbf{GENERICS LOCALI (Polimorfismo parametrico di primo ordine)} 

\begin{lstlisting}[basicstyle=\small,]
    public static Object ident__ugly(Object o) {
        return o;
    }   
    /* con un metodo di subtyping che è POLIMORFISMO VERTICALE, 
     * questa funzione NON è SOUND perché sono costretto a 
     * fare un CAST di ciò ciò che ricevo */

    public static <X> X ident(X x) {
        return x;
    }   
    /* con i generics, che è POLIMORFISMO PARAMETRICO, non
     * devo più fare nessun cast. Sono sicuro del tipo di
     * ritorno */
\end{lstlisting}

\begin{lstlisting}[basicstyle=\small,]
	public static void main__ident	(){
		Cane fido = new Cane();
		Cane c = (Cane) ident__ugly(fido); /* ritorna un cane 
											* facendo un cast */
		Cane c2 = ident(fido);  /* ritorna un cane senza dover 
								 * fare un fast */ 
		Gatto g = ident(new Gatto());
	}
\end{lstlisting}





















