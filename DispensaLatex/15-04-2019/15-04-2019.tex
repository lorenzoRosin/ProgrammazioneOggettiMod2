

\newpage
\section{15-04-2019: THREAD POOL}
\noindent \textbf{APPUNTI PRESI IN CLASSE} \newline
Nel codice visto a lezione abbiamo usato i seguenti import :
\begin{lstlisting}
import org.jetbrains.annotations.NotNull;
import org.jetbrains.annotations.Nullable;
\end{lstlisting}
Cosi facendo abbiamo accesso a delle nuove features! \newline
\textbullet\ nullness: \textit{@Nullable} con questa annotazione si possono marcare i metodi, campi e variabili per suggerire che quell'entità può ritornare null come valore. Inoltre serve a rimarcare che bisogna controllare che il valore che viene restituito da una funzione sia null. \newline
\textbullet\ notnullness: \textit{@NotNull} con questa annotazione assicuriamo che l'elemento non sia mai null. \newline

\noindent \textbf{CODICE DEL PROFESSORE} \newline
\noindent Ci sono due implementazione della classe ThreadPool. \newline

\begin{lstlisting}
/* Classe: PoolMain.java */
/* Usa il ThreadPool presente in
 * TINIJDK/threads */
package threads;

import java.util.Random;
import java.util.concurrent.TimeoutException;

public class PoolMain {

    public static void main(String[] args) {
        ThreadPool pool = new ThreadPool(10, 3);
        while (true) {
            ThreadPool.print("%s", pool);
            try {
                Thread t = pool.acquire(() -> {
                    final long now = System.currentTimeMillis();
                    sleep(500);
                    ThreadPool.print("took %d ms", System.currentTimeMillis() - now);
                });
                sleep(100);
            } catch (InterruptedException | TimeoutException e) {
                e.printStackTrace();
            }
        }
    }

    private static Random rand = new Random();

    private static void sleep(int ms) {
        try {
            Thread.sleep(rand.nextInt(ms));
        } catch (InterruptedException e) {
            e.printStackTrace();
        }
    }
}
\end{lstlisting}



\begin{lstlisting}
/* Classe: ThreadPool.java */
/* Ha modificato la classe che
 * stava in TINJJDK/threads */

package threads;

import org.jetbrains.annotations.NotNull;
import org.jetbrains.annotations.Nullable;

import java.util.Objects;
import java.util.concurrent.BlockingQueue;
import java.util.concurrent.LinkedBlockingQueue;
import java.util.concurrent.TimeUnit;
import java.util.concurrent.TimeoutException;

public class ThreadPool {
    private static final long DEFAULT_ACQUIRE_TIMEOUT_MS = 500L;
    @NotNull
    private final BlockingQueue<PooledThread> q;
    private final int max;
    private int total = 0;

    @Override
    public String toString() {
        return String.format("ThreadPool[available:%d total:%d max:%d]", available(), total(), max);
    }

    private synchronized void incrementTotal() {
        ++total;
    }

    private synchronized void decrementTotal() {
        --total;
    }

    public ThreadPool(int max, int prespawn) {
        q = new LinkedBlockingQueue<>(max);
        this.max = max;
        for (int i = 0; i < Math.min(max, prespawn); ++i) {
            q.add(spawn());
        }
    }

    public ThreadPool(int max) {
        this(max, 0);
    }

    private static void print(String s) {
        System.out.println(String.format("%s: %s", Thread.currentThread(), s));
    }

    public static void print(String fmt, Object... o) {
        print(String.format(fmt, o));
    }

    protected class PooledThread extends Thread {
        @Nullable
        private Runnable currentRunnable;

        @Override
        public String toString() {
            return String.format("[%s:%d]", getName(), getId());
        }

        @Override
        public void run() {
            incrementTotal();
            synchronized (ThreadPool.this) {
                total++;
            }

            try {
                while (true) {
//                    print("entering wait state...");
                    @NotNull Runnable r;
                    synchronized (this) {
                        wait();
                        r = Objects.requireNonNull(currentRunnable);
//                        print("awaken: executing custom runnable...");
                    }
                    try {
                        r.run();
                    } catch (Throwable e) {
                        print("exception caught during execution of custom runnable");
                        e.printStackTrace();
                    }
                    synchronized (this) {
                        currentRunnable = null;
                    }
                    if (!q.offer(this)) {
                        print("cannot requeue due to space limit");
                        break;
                    }
                }
            } catch (InterruptedException e) {
                print("interrupted exception");
                e.printStackTrace();
            }
            print("exiting");
            decrementTotal();
        }
    }

    public int available() {
        return q.size();
    }

    public synchronized int total() {
        return total;
    }

    @NotNull
    private PooledThread spawn() {
        PooledThread p = new PooledThread();
        p.start();
        print("spawned new thread %s", p);
        return p;
    }

    @NotNull
    public Thread acquire(@NotNull Runnable cb, long time, @NotNull TimeUnit unit) throws InterruptedException, TimeoutException {
        @NotNull final PooledThread p;
        if (q.isEmpty()) {
            if (total() < max)
                p = spawn();
            else {
                print("pool max reached: waiting for an available thread...");
                PooledThread tmp = q.poll(time, unit);
                if (tmp == null)
                    throw new TimeoutException();
                p = tmp;
            }
        } else {
            p = q.poll();
        }
        synchronized (p) {
            p.currentRunnable = cb;
            p.notify();
        }
        return p;
    }

    @NotNull
    public Thread acquire(@NotNull Runnable cb) throws InterruptedException, TimeoutException {
        return acquire(cb, DEFAULT_ACQUIRE_TIMEOUT_MS, TimeUnit.MILLISECONDS);
    }

}


\end{lstlisting}
 
\begin{lstlisting}
/* Classe: ThreadPool.java */
/* Ha modificato la classe che
 * stava in work */

package threads.work_in_progress;

import org.jetbrains.annotations.NotNull;
import org.jetbrains.annotations.Nullable;

import java.util.Objects;
import java.util.Queue;
import java.util.concurrent.ConcurrentLinkedQueue;

public class ThreadPool {
    private Queue<PooledThread> q = new ConcurrentLinkedQueue<>();

    public static class PooledThread extends Thread {
        @Nullable
        private Runnable currentRunnable;

        @Override
        public void run() {
            while (true) {
                try {
                    synchronized (this) {
                        print("waiting...");
                        wait();
                        print("awaken");
                    }
                    Objects.requireNonNull(currentRunnable).run();
                } catch (InterruptedException e) {
                    e.printStackTrace();
                }
            }
        }

        @Override
        public String toString() {
            return String.format("[%s:%d]", this.getName(), this.getId());
        }

    }

    public PooledThread acquire(Runnable cb) {
        @NotNull final PooledThread p;
        if (q.isEmpty()) {
            System.out.println("spawning");
            p = new PooledThread();
            p.start();
        } else {
            p = q.poll();
        }
        synchronized (p) {
            print("notify");
            p.currentRunnable = cb;
            p.notify();
            return p;
        }
    }

    public void release(PooledThread t) {
        q.add(t);
    }


    public static void main(String[] args) {
        ThreadPool pool = new ThreadPool();
        Thread t1 = pool.acquire(() -> {
            try {
                Thread.sleep(300);
            } catch (InterruptedException e) {
                e.printStackTrace();
            }
            print("bye");
        });
    }

    private static void print(String s) {
        System.out.println(String.format("%s: %s", Thread.currentThread(), s));
    }
} 

\end{lstlisting}