

\newpage
\section{APPROFONDIMENTI}
In questa sezione inseriro il materiale utile trovato nel libro e/o su internet sotto forma di approfondimento. 


\noindent \textbf{COSA STUDIARE BENE} \newline
Per l'esame bisogna studiare: \newline
\textbullet\ Collection \newline
\textbullet\ Thread \newline
\textbullet\ ThreadPool \newline
\textbullet\ Design pattern: factory \newline
\textbullet\ Design pattern: singleton \newline
\textbullet\ Design pattern: command \newline
\textbullet\ Programmazione funzionale: lambda \newline
\textbullet\ Programmazione funzionale: anonymus classes \newline
\textbullet\ Programmazione funzionale: function objects \newline


\subsection{COLLECTION}
\noindent Le collection le chiamano anche contenitori.


\subsection{FACTORY}


\subsection{COMMAND}
Questo design pattern prevede di fare un decoupling tra un attore che produce azioni ed un consumatore.



\subsection{FUNCTION OBJECT}
\noindent Sono oggetti usati solo come funzioni. Essi magari implementano un'interfaccia funzionale e poi vengono passati nel codice ed usati solo per il loro unico metodo. Questi oggetti non si memorizzano variabili interne ma elaborano solamente i dati in ingresso al loro metodo. vediamone un esempio:
\begin{lstlisting}
	public interface Function{
		public int exec(int param);
	}
	
	public class SumTwo implements Function{
		@Override
		public int exec(int param){
			return param+2;
		}
	}
	
	public static void main(String[] strings){
		SumTwo obj = new SumTwo();
		System.out.println(obj.exec(10));
	}
\end{lstlisting}


