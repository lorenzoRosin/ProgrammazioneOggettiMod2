

\newpage
\section{11-02-2019}
\par

\textit{ITERATORE}: E' un pattner, uno stile di programmazione. Il pattern degli iteratori esiste in tutti i linguaggi ad oggetti. Con iteratore intendiamo lo scorrimento di una collezione di elementi. \newline
\textit{ITERABLE} è una super interfaccia, e l'interfaccia \textit{COLLECTION} implementa questa super interfaccia. Iterable è super tipo di tutte le interfacce. \newline
<? extends E> \newline
\textit{SOTTOTIPO} = 1) sei una sottoclasse (extends) 2) sei una sottointerfaccia (implements)
\newline
La \textit{SUBSUMPTION} non funziona tra GENERICS. Per il parametro stesso c'è subsumption, ma non per le collection.
\newline
TIPO ESTERNO: funziona sempre la subsumption \newline
TIPO INTERNO: non funziona, solo con <? extends E> \newline
[Invarianza del subtyping]: Se ciò non fosse le assunzioni funzionerebbero anche nel tipo di ritorno e questo rischierebbe la totale spaccatura \newline
Se cosi non fosse in java non verrebbero mai rispettate le regole delle classi. \newline
Java di unico ha che esiste il wildchart (?), che è un modo controllato per risolvere questo problemino. \newline
Prima dei generics (2003/2004) in java si programmava tutto a typecast. Per motivi di retrocompatibilità è possibile programmare in tutti e due i modi. E' comunque consigliato usare la porgrammazione con i GENERICS. \newline
Metodi che ritornano un booleano iniziano con sempre come se fossero domande; es: hasNext, isEmpty etc.. \newline
Un iteratore non può essere costruito con un new perchè è un'interfaccia. 



\newpage
