\documentclass[a4paper,4pt]{article}
\usepackage[utf8]{inputenc}
\usepackage{graphicx}
\usepackage{sidecap}
\usepackage{amsmath}
\usepackage[T1]{fontenc}
\usepackage[italian]{babel}
\usepackage{geometry}
\usepackage{booktabs}
\usepackage{color}
\usepackage{xcolor}
\usepackage{gensymb}
\usepackage{colortbl}
\usepackage{wrapfig}
\usepackage{floatflt,epsfig}
\usepackage[]{subfig}
\usepackage[]{caption}
\usepackage{listings}
\usepackage[hidelinks]{hyperref}
\usepackage[pdf]{graphviz} 
\graphicspath{{7-02-2019/},{4-02-2019/},{11-02-2019/}}
\setlength{\parskip}{10pt plus 1pt minus 1pt}
\definecolor{xanadu}{rgb}{0.45, 0.53, 0.47}
\usepackage{amssymb}
\lstset{
    language=JAVA, %% Troque para PHP, C, Java, etc... bash é o padrão
    basicstyle=\ttfamily\small,
    numberstyle=\footnotesize,
    numbers=left,
    backgroundcolor=\color{gray!10},
    frame=single,
    tabsize=2,
    rulecolor=\color{black!30},
    title=\lstname,
    escapeinside={\%*}{*)},
    breaklines=true,
    breakatwhitespace=true,
    framextopmargin=2pt,
    framexbottommargin=2pt,
    inputencoding=utf8,
    extendedchars=true,
    showstringspaces = false,
    literate={á}{{\'a}}1 {ã}{{\~a}}1 {é}{{\'e}}1 {è}{{\'e}}1 {ò}{{\'o}}1  {ù}{{\'u}}1 {à}{{\'a}}1 ,   
    commentstyle=\color{xanadu},
}

\begin{document}



\tableofcontents


\newpage
\section{4-02-2019: INTRO}
\textbf{DICHIARAZIONE $\neq$ ASSEGNAMENTO} \newline
L'assegnamento fa riferimento alla modifica di una variabile.
\begin{lstlisting}
int n		   /* Dichiarazione */
int m = n = 1; /* Inizializzazione */
n = 2 		   /* Assegnamento */
\end{lstlisting}

\noindent \textbf{PROGRAMMAZIONE IMPEATTIVA VS FUNZIONALE} \newline
JAVA è un linguaggio \textit{imperattivo} ad oggetti. Questo significa che si programma variando lo stato interno delle varie istanze. In contrapposizione troviamo la programmazione funzionale che si basa sul richiamare e scrivere funzioni che non variano gli stati interni delle istanze. Nella programmazione funzionale sono usati spesso metodi che ritornano copie modificate di un determinato oggetto, senza modificare l'oggetto attuale.

\noindent \textbf{JAVA VIRTUAL MACHINE} \newline
\noindent Java è stato realizzato con un compilatore integrato, che non compila in assembly, questo compilatore invece produce un file sorgente che non è eseguibile direttamente dalla macchina, bensi è eseguibile da una virtual machine, la JVM(\textit{java virtual machine}). In questo modo viene garantita la portabilità del codice in vari computer con CPU diversa (come il Phyton e .NET). \newline
Questa separazione non conta niente per il linguaggio, si riflette solamente sul modello architetturale.

\noindent U.M.L. = unified model language $\Rightarrow$ rappresentazioni di gerarchie di classi. \newline
\digraph{ao}{rankdir=LR;

   a [label="Animali" shape = "record"]; 
   c [label="Cani" shape = "record"]; 
   g [label="Gatti" shape = "record"]; 
   d [label="Dalmata" shape = "record"]; 
   c->a; g->a; d->c;} \newline
\textbullet\ Tutte le sottoclassi sono dei sottoinsiemi. \newline
\textbullet\ Tutte le superclassi sono dei sovrainsiemi. 

\noindent I linguaggi ad oggetti ci permettono di costruire \textit{tipi} e di definire \textit{valori}. 
\begin{lstlisting}
	Animale a = new Animale();
	/* in Java gli oggetti sono valori
	 * dove: Animale -> tipo -> CLASSE
	 * a = nome variabile
	 * new Animale() ha un valore -> OGGETTO */
\end{lstlisting}

\noindent \textbf{COMPILING TIME E RUNTIME} \newline
La gestione di sintassi e di errori viene fatta in due fasi.
Dal compilatore(compiling time): \newline
\textbullet\ Viene eseuito un controllo sintattico \newline
\textbullet\ Viene eseguito controllo dei tipi, cioè che gli insiemi siano corretti\newline
In fase di esecuzione(RunTime):\newline
\textbullet\ Abbiamo a che fare con valori e non con tipi, vengono controllati i "cast"

\noindent I tipi di fatto sono una astrazione del linguaggio. \newline
Il senso di un compilatore è quello di evitare di scrivere castonerie che a livello di esecuzione non avrebbero senso.

\noindent \textbf{SOUNDNESS} \newline
Un linguaggio si dice \textit{sound} quando il compilatore ti da la certezza che in fase di esecuzione il programma sia eseguito correttamente senza possibilità di errori.

\noindent \textbf{PARAMETRO IMPLICITO} \newline
 Ogni metodo dichiarato ha sempre un parametro implicito (il parametro \textit{this}). Esso è sotto inteso e viene passato automaticamente, più eventuali parametri che vengono passati all'interno del metodo.
\begin{lstlisting}
	public class Animal {
		private int peso;
		...
		public void mangia(Animali a){
			this.peso = this.peso + a.peso;
		}
	}
	
	Cane fido = new Cane();
	a.mangia(fido); 			/* SUBSUMPTION (assunzione)	*/
\end{lstlisting}

\noindent \textbf{POLIMORFISMO} \newline
L'eredità è un meccanismo che garantisce il funzionamento del \textit{polimorfismo}.\newline
\textbullet\ Polimorfismo per \textit{SUBTYPING} o anche polimorfismo per inclusione: si riferisce al fatto che una espressione il cui tipo sia descritto da una classe A può assumere valori di un qualunque tipo descritto da una classe B sottoclasse di A (Vedi codice appena sopra)\newline 
\textbullet\ Polimorfismo dei \textit{GENERICS}: si riferisce al fatto che il codice del programma può ricevere un tipo come parametro invece che conoscerlo a priori (polimorfismo parametrico). In questo modo non si perde il tipo originario passato dall'oggetto al metodo \newline
\begin{lstlisting}
	/* POLIMORFISMO SUBTYPING
	 * basato sui sottotipi ereditarietà */
	Object ident (Object x){
		return x;
	}

	/* POLIMORFISMO GENERICS (parametrico)
	 * non perdo informazioni sui tipi */
	<T> T ident (T x){
		return x;
	}
	/* questa funzione mi permette di riusare il 
	 * metodo, in questo modo evito di fare CAST,
	 * e di sbagliare a farli */
\end{lstlisting}

\newpage




\newpage
\section{7-02-2019: DINAMIC DISPATCHING}
Nei linguaggi ad oggetti, lo strumento più potente è la classe. Quando definisco le entità che poi vado a tramutare in classi sto definendo DATI. \newline
Le classi possono contenere dei metodi (funzioni che operano sugli oggetti della classe). \newline
Definire sottoclassi significa definire \textit{sottoinsiemi} nell'ambito dell'ereditarietà. Le nuove operazioni delle sottoclassi vanno inserite sapendo che le sottoclassi ereditano il set di funzioni delle superclassi. 
\textit{IL POLIMORFISMO} è uno strumento molto utile perchè ci permette di scrivere codice, funzioni che posso adoperare anche con tipi diversi! \newline
@ serve per creare delle annotazioni nel codice, serve per il compilatore (es: @ override).

\noindent \textbf{OVERRIDING} \newline
L'\textit{Overriding} è il punto cruciale di tutta la programmazione ad oggetti. Fare overriding significa sovrascrivere un metodo ereditato dalla super classe per poterne specializzare il suo comportamento. Se non potessi farlo significa che nelle sottoclassi non posso andare a specializzare un metodo. Specializzare un metodo significa cambiare l'implementazione della super classe senza cambiarne la firma. 


\noindent \textbf{DINAMIC DISPATCHING} \newline
Il \textit{Dinamic dispatching} serve in fase di runtime a scegliere la versione giusta del metodo da richiamare. Infatti se ho degli over ride nelle mie classi, sarà solo in fase di run time che Java deciderà quale metodo richiamare. Se nella mia classe non esiste il metodo richiamato, il dinamic dispatching va a prendere l'implementazione del metodo dalla superclasse. Nella memoria che contiene le informazioni degli oggetti ci sono tutti i puntatori ai metodi di una classe, in run time viene eseguito il codice del puntatore corretto. (\textit{vedere: virtual table}).
Un \textit{OGGETTO} infatti è costituito da un insieme dei suoi campi e da puntatori ai metodi della classe ed è grazie a questo che il dispatching funziona: il compilatore controlla i tipi e garantisce che nel compiling time tutto questo funzioni. 

\noindent Ogni espressione ha un tipo!

\noindent \textbf{CLASSI E METODI STATICI} \newline
\textbullet\ I metodi statici sono quei metodi di una classe che appartengono alla classe, non alle istanze di una classe. Si possono richiamare senza creare un oggetto. I metodi statici possono quindi accedere solo a dati statici e non alle variabili di istanza della classe, possono solo richiamare altri metodi statici della classe, e sopratutto non possono usare il parametro implicito \textit{this}. \newline
\textbullet\ Le classi statiche in java possono solamente esistere se sono \textit{innestate (nested)}. Esse possono accedere solamente dati statici della classe che le contiene. Una classe statica interna non vede il riferimento this dell'altra classe, essa può accedere solamente ai campi statici della classe che la contiene.\newline
Le classi statiche non sono però come i membri statici, è possibile infatti instanziarne più istanze, tutte quante indipendenti e non relazionate tra di loro. 

\noindent \textbf{COLLECTION} \newline
Le \textit{Collection o contenitori} sono delle interfacce della libreria di JAVA e non si possono costruire.\newline
Le \textit{Collection} da sole non sono dei tipi, le \textit{Collection} di un "qualcosa" sono dei tipi.I tipi parametrici vogliono infatti un \textit{argomento} \newline
\begin{center}
\includegraphics[width=%
0.6\textwidth]{java-collection-hierarchy}
\end{center} 

\noindent \textbf{PACCHETTI JAVA}\newline
JAVA SE $\Rightarrow$ Standard Edition \newline
JAVA EE $\Rightarrow$ Enterprise Edition \newline
JAVA ME $\Rightarrow$ Mobile Edition \newline
JAVA JDK $\Rightarrow$ linguaggio + tutte le librerie standard (java developement kit) \newline
JAVA JRE $\Rightarrow$ Solo a runtime, versione ridotta che serve solo a chi usa i programmi ma non al programmatore (java runtime enviroment) \newline
File jar  $\Rightarrow$ Archivio di tutti i pacchetti del programma \newline
JAVA JVM $\Rightarrow$ (Java virtual machine) serve per eseguire i file .jar \newline
La documentazione di java si trova on-line ed è diffusa in pacchetti che servono ad organizzare logicamente le classi, che sono organizzate in ordine alfabetico. \newline





\newpage













\newpage
\section{11-02-2019}
\par

\textit{ITERATORE}: E' un pattner, uno stile di programmazione. Il pattern degli iteratori esiste in tutti i linguaggi ad oggetti. Con iteratore intendiamo lo scorrimento di una collezione di elementi. \newline
\textit{ITERABLE} è una super interfaccia, e l'interfaccia \textit{COLLECTION} implementa questa super interfaccia. Iterable è super tipo di tutte le interfacce. \newline
<? extends E> \newline
\textit{SOTTOTIPO} = 1) sei una sottoclasse (extends) 2) sei una sottointerfaccia (implements)
\newline
La \textit{SUBSUMPTION} non funziona tra GENERICS. Per il parametro stesso c'è subsumption, ma non per le collection.
\newline
TIPO ESTERNO: funziona sempre la subsumption \newline
TIPO INTERNO: non funziona, solo con <? extends E> \newline
[Invarianza del subtyping]: Se ciò non fosse le assunzioni funzionerebbero anche nel tipo di ritorno e questo rischierebbe la totale spaccatura \newline
Se cosi non fosse in java non verrebbero mai rispettate le regole delle classi. \newline
Java di unico ha che esiste il wildchart (?), che è un modo controllato per risolvere questo problemino. \newline
Prima dei generics (2003/2004) in java si programmava tutto a typecast. Per motivi di retrocompatibilità è possibile programmare in tutti e due i modi. E' comunque consigliato usare la porgrammazione con i GENERICS. \newline
Metodi che ritornano un booleano iniziano con sempre come se fossero domande; es: hasNext, isEmpty etc.. \newline
Un iteratore non può essere costruito con un new perchè è un'interfaccia. 



\newpage


 
\newpage
\section{15-02-2019}
\textbf{INTERFACCE}\newline
\begin{lstlisting}[basicstyle=\small,]

	public interface Iterator<T>{}

\end{lstlisting}
L'interfaccia è un contratto, nel senso che mette a disposizione una serie di metodi che ogni classe che estende quell'interfaccia deve obbligatoriamente implementare, pena un errore durante la fase di compilazione.In java quindi si può scrivere del codice ancora prima di sapere come si potrebbe implementare.

\noindent Facciamo un esempio: il "contratto" di iteratore è il seguente: \newline
\textbullet\ boolean hasNext(); \newline
\textbullet\ T next(); \newline
\textbullet\ void remove()\newline
Data una certa classe che può non essere sotto al nostro controllo non abbiamo bisogna sapere necessariamente come sono stati implementati i suoi metodi, ma ci ti basta sapere che esistono per poter dire se sia o meno un iteratore. \newline
Esempio di definizione di un metodo con iteratore come input: 
\begin{lstlisting}[basicstyle=\small,]

	public statics void scorri(Iteratore <Integer> it){
		while(it.hasnext()){
			integer n = it.next();
		}
	}

\end{lstlisting}
Esempio di utilizzo 
\begin{lstlisting}[basicstyle=\small,]

	scorri(new Iterator<>(){
		...
		...
		...
	});

\end{lstlisting}
Quest'ultima è un'espressione, o come meglio dire, un'oggetto fatto al volo. Questa sintassi è stata creata appositamente per le interfacce (dato che non si possono istanziare direttamente), senza dover andare a definire una classe con la classica implementazione dell'interfaccia. 

\noindent \textit{ANONYMOUS CLASS} meccanismo comodo per design pattern come le call-back. 

\noindent Questa implementazione garantisce che la funzione sia \textit{SOUND}, e non crasherà mai a \textit{RunTime}

\noindent \textbf{IMPLEMENTARE INTERFACCE}\newline
1) Con implements: \newline
\textbullet\ controlla i metodi che hai implementato all'interno della classe \newline
\textbullet\ assicura che siano implementati tutti 

\noindent Tipi delle interfacce \newline
Iterator $\Rightarrow$ non è un tipo \newline
Iterator<T> $\Rightarrow$ è un tipo 

\noindent \textbf{NOTAZIONE BNS} \newline
BNS è il nome della notazione e serve per poter dare delle regole grammaticali. E' una notazione che definisce la sintassi delle espressioni\newline
Iterator da solo, sintatticamente, sarebbe un tipo. Ma il compilatore verifica che non è un tipo e da errore.








\newpage 




\newpage
\section{18-02-2019}
\par

una è \textit{CLASSE ASTRATTA} quando ha almeno un metodo astratto, serve per impedire la sua costruzione (non ne posso costruire quindi una istanza) e vengono definite astratte se  anche un solo metodo è astratto. \newline
Un array è una struttura dati lineare, omogenea e contigua in memoria. \newline
Una interfaccia è zucchero sintattico di una classe astratta con soli metodi astratti. \newline
La differenza tra classe ed interfaccia non esiste. \newline
Per leggere una collection si usano gi iteratori che servono per gettare in sequenza.

\begin{lstlisting}[basicstyle=\small,]

public static class Animale(){
	privae int peso;
}

public static class Cane extends Animale{
	private String nome;
	public void abbaia(){};
}

public static class PastoreTedesco extends Cane{

}

\end{lstlisting}

Se costruisco un oggetto di tipo pastoreTedesco, esso sarà grande quanto un tipo int (32 bit) ed una stringa (un puntatore).
Il tutto grazie alla virtual table che tiene in memoria i puntatori dei vari campi di uno oggetto.









\newpage
\section{21-02-2019}
\par

\textit{REFLECTION} è una tecnica per conoscere i tipi e il contenuto delle classi  a runtime. \newline
Se voglio conoscere il tipo dell'enclosing class (classe che contiene) posso fare : nome.classe.this.nome \newline

\textit{BINDING} avviene anche con i tipi \newline
I parametri di una funzione sono binding nello scope della funzione.\newline
I type argument fanno binding con i type parameter, esattamente come avviene per le funzioni tra argomenti e parametri. \newline

Quando si programma con i generics si PROPAGANO. \newline

TYPE ERASURE: cancellazione dei tipi: java lo fa quando compila butta via i generics generando classi non anonime e li sostituisce con Object: il motivo è per mantenere la compatiblità con il vecchio codice che non aveva generics. Quindi i generics sono verificati dal compilatore e poi cancellati per eseguire.

















\newpage
\section{25-02-2019: EREDITARIETA'}
Se un oggetto ha dei campi esso pesa tanto quanto la dimensione dei campi.Ricordiamo che nei pc a 64 bit i puntatori pesano 8 byte.

\noindent L'ereditarietà serve anche a modificare i metodi della classe che viene ereditata. E' l'unico modo che abbiamo per modificare delle cose anche se non sappiamo cosa e chi le ha costruite. Sopratutto se non le possediamo. Un esempio è la classe ArrayList, che deve essere ereditata per implementare un motodo che ci permetta di scorrerla all'indietro. \newline
I metodi statici non si possono override perchè non sono presenti nelle virtual table (sono funzioni sciolte). \newline

Regola ereditarietà costruttore: se non definisco nessun costruttore nella sottoclasse è come se chiamassi il costruttore della superclasse SENZA parametro. 









\newpage
\section{28-02-2019}
\par

\textit{COVARIANZA e CONTROVARIANZA} dei tipi \newline
$C_{1} <\tau_{1}> \leq C_{2}<\tau_{2}> \Leftrightarrow C_{1} \leq C_{2} \bigwedge \tau_{1}\equiv \tau_{2} $

Questa regola del type system di java si dice che il linguaggio NON è COVARIANTE, in quanto i generics non cambiano. \newline

\begin{lstlisting}[basicstyle=\small,]

@Override
public PastoreTedesco m(Animale c){ return new PastoreTedesco;}
\\ il tipo di ritorno del metodo (PastoreTedesco) scende (sottoclassi)
\\ il tipo del parametro di ingresso (Animale) sale (superclasse)

\end{lstlisting}

In java è possibile controvariare solo il tipo di ritorno del metodo, solo scendendo (sottotipo) \newline


\begin{lstlisting}[basicstyle=\small,]

public Cane m (Cane c){return c;}

@Override
public PastoreTedesco m(Cane c){return new PastoreTedesco();}

\end{lstlisting}


SOUND: un programma che compila e termina, a meno di una eccezione. \newline
In java è possibile avvenga un segmentation fault non per un problema di casting, ma solamente se accediamo ad un indice di un array non abbiamo allocato. \newline




Ci sono linguaggi dove non esistono gli array, quindi non accadrà mai segmentation fault e il codice terminerà sempre, ovviamente senza fare i controlli di semantica. \newline
Recentemente è stato inserito un pattern che qualcosa la covarianza: \newline


\begin{lstlisting}[basicstyle=\small,]

Arraylisti<? extends Animale> m = new Arraylist<Cane>();

\end{lstlisting}

Da questo si capisce che la covarianza può essere usata, ma solamente se esplicitata con il wildcard. \newline
Sono molto usati perchè non sono tipi del primo ordine \newline
Non posso definire una variabile: \newline

\begin{lstlisting}[basicstyle=\small,]
? extends Animale m = new Cane();
\\ questa sintassi si puo usare solo come type argument
\end{lstlisting}

Significato: permettono la covariamza, sono tipi temporanei che non possono essere scritti nel codice, però possono essere sostituiti con il get(). \newline
Un altro DESIGN PATTERN: callback

















\newpage
\section{4-03-2019}
\textbf{DESIGN PATTERN} \newline
\textbullet\ Iteratore \newline
\textbullet\ Compact o callback ounary function 

\noindent Le \textit{LAMBDA ASTRAZIONI} servono per fare funzioni al "volo", senza dover implementare in classi separate delle interfacce	

\noindent \textbf{FUNZIONI DI ORDINE SUPERIORE} \newline
Sono delle funzioni che prendono delle funzioni come parametri di ingresso 
\begin{lstlisting}[basicstyle=\small,]

\\ questa interfaccia essere equivalente alla interfaccia java.util.Functional
public interface Func<A, B>{
	B execute(A a); \\ questa essere l unica funzione esposta dalla interfaccia
	\\un altro nome ragionevole per il metodo execute() essere apply() oppure call()
	\\il nome deve ricordare il fatto di richiamare la funzione 
}

public static <A,B> List<B> map(List<A> I, Func<A,B> f){
	List<B> r = new ArrayList<>();
	for(A x: l)
		r.add(f.execute(x));
	return r;

}

\end{lstlisting}
A e B sono \textit{generics} locali al metodo(e solo al metodo) \newline
I generics sulle classi servono per parametrizzare, non per fare polimorfismo \newline
\begin{lstlisting}[basicstyle=\small,]

public static <A,B> List<B> map(List<A> I, Func<A,B> f){
\\ dove in public static <A,B> dichiaro i parametri che useremo
\\ mentre in List<a> .. Func <A,B> "uso" i parametri

\end{lstlisting}

\noindent Funzione FILTER: 
\begin{lstlisting}[basicstyle=\small,]

public static <A> List<A> Filter (List<A> l, Func<A,Boolean> p){
	List<A> r = new ArrayList<>();
	for(A x : l)
		if(p.execute(x))
			r.add(x);
	return r;
}

\end{lstlisting}

\noindent La seguente funzione NON funziona perché usa la remove() delle Collection, ma non è possibile rimuove un elemento in fase di scorrimento (è scritto nella documentazione)

\begin{lstlisting}[basicstyle=\small,]

public static <A> void Filter2 (List<A>, Func<A, boolean> p){
	for(A a: l)  //il for each in Java essere zucchero sintattico
		if(p.execute(a))
			l.remove(a);

}

\end{lstlisting}
Se non posso rimuovere come ho fatto sopra un elemento posso invece chiedere all'iteratore di rimuovere l'elemento stesso, esso rimuoverà quello a cui stiamo puntando

\begin{lstlisting}[basicstyle=\small,]

// questo funziona perche chiama la remove() dell'iteratore
public static <A> void Filter2 (List<A>, Func<A, boolean> p){
	Iterator<A> it = l.iterator();
	while(it.hasNext()){
		A a = it.next();
		if(!(p.execute(a)))
			it.remove();
	}
}

\\volendo posso usare le funzionalita delle nuove API FUNZIONALI
\\l.removeIf(a -> !p.execute(a));
\end{lstlisting}

\noindent Posso usare Function<A,B> di java come funziona func? \newline
Esempio di chiamata:

\begin{lstlisting}[basicstyle=\small,]

public static void main(){
	List<String> strings = new ArrayList<>();
	string.add("ciao");
	string.add("pippo");
	string.add("unive");
	List<Integer> r = map(strings, new Func<String, Integers>{
		@Override 
		public Integer execute(String a){
			return a.length();
		}	
	});
}
\end{lstlisting}


\noindent La seguente funzione data una lista di interi scarta gli elementi minori di zero: questo è il modo per non usare un for con un ciclo if innestato.

\begin{lstlisting}[basicstyle=\small,]

public static void main__filter(){
	 List<Integer> interi  = new ArrayList<>();
	 interi.add(89);
	 interi.add(34);
	 interi.add(-16);
	 interi.add(560);
	 interi.add(-1);
	 interi.add(46);
	 //filter prende una lista e un predicato e produce una lista in uscita
	 List<Integer> l = Filter(ints, new Func<Integer, boolean>(){
	 	@Override
	 	public Boolean execute (Integer a){
	 		return a>=0;
	 	}
	 });
}
\end{lstlisting}

Oppure 

\begin{lstlisting}[basicstyle=\small,]
	Filter2 (interi , new Func<Integer, Boolean>(){
		@Override
		public Boolean execute (Integer a)
			return a>=0;
	})
\end{lstlisting}

\noindent \textbf{Generics Locali (Polimorfismo parametrico di primo ordine)} 

\begin{lstlisting}[basicstyle=\small,]

    public static Object ident__ugly(Object o) {
        return o;
    }   
    //con metodo di subtyping che è POLIMORFISMO VERTICALE, 
    //questa funzione NON è sound perché sono costretto a castare ciò che ricevo

    public static <X> X ident(X x) {
        return x;
    }   //con i generics che è POLIMORFISMO PARAMETRICO

\end{lstlisting}

\begin{lstlisting}[basicstyle=\small,]

	
	public static void main__ident	(){
		Cane fido = new Cane();
		Cane c = (Cane) ident__ugly(fido);  //ritorna un cane castando
		Cane c2 = ident(fido);  //ritorna un cane senza dover castare
		Gatto g = ident(new Gatto());
	}
\end{lstlisting}
























\newpage
\section{7-03-2019}
\par




\begin{lstlisting}[basicstyle=\small,]

List<?> l1 = new ArrayList<Cane>();
\\ ? -> Da solo significa che indica un tipo che gerarchicamente
\\ e pie in alto di Object, viene detto top type

l1.get(int index) \\ritorna un capture of ? , qualcosa che sia figlio del top type 

\end{lstlisting}

il tipo ? non può essere usato come tipo per una variabile, ? x non si può fare, però posso fare
 Object x = l1.get(..) \newline
 
Mentre ? extends Animale -> qualsiasi cosa che sia figlio di animale \newline
l2.get(0) -> ritorna un capture of ? extends Animale -> qualsiasi cosa figlia di animale (posso però fare Binding di qualcosa che sia al massimo Animale) \newline

Posso subsumero solo il tipo esterno, se voglio subsumere anche il tipo interno devo usare le wildcards. \newline
? super Animale -> qualcosa che sia più su di Animale (più generale) \newline
l2.add(new Animale) -> ?? non compila perchè... \newline

Riprenderemo la map vista l'altra volta. Ad esempio, per trasformare animali in piante: 



\begin{lstlisting}[basicstyle=\small,]

public static class Vegetale{}

public static void main_map(){
	List<cani> l1 = new ArrayList<>();
	List<Vegetali> l2 = map(l1, new func<Animale, Vegetale>(){
		@Override
		public Vegetale execute(Animale a){
			return null;
		}
	
	});
}

\end{lstlisting}

Questo non compila in quanto i generics non sono soggetti alla subsumption. \newline
Per farlo compilare modifichiamo la funzione map: 


\begin{lstlisting}[basicstyle=\small,]


public static  <x, y> List<x> map(List<x>, Func(? super <x, y> f)){
...

}

\end{lstlisting}



NESTED CLASS \newline
La Nested Class è totalmente senza relazione rispetto alla enclosed class.\newline
Nel caso precedente main{\_}functional è la enclosed class, in quanto sto lavorando su quella. \newline
Le nested class vedono i campi della enclosing class, ma se sono statiche non vedono i campi. 

\begin{lstlisting}[basicstyle=\small,]


public static  <x, y> List<y> map(List<x>, Func<? super x ,? extends y> f)){
...

}

\end{lstlisting}
Questa è la versione più generale possibile. \newline

Overloading \newline
Permette di definire metodi con stesso nome ma firma diversa. 

\begin{lstlisting}[basicstyle=\small,]


public static class c{
	public int m(){
		return 1;
	}
	public int m(int x){
		return x+1;
	}
	public int m(float x){
		return (int)(x-1.0f);
	}
	public int m(int x, int y){
		return x+y;
	}		
}

\end{lstlisting}

L'overloading non è permesso cambiando il tipo di ritorno e lasciando il resto inalterato. Devono essere diversi i parametri! \newline
-ordine \newline
-tipi \newline
-numeri \newline


\begin{lstlisting}[basicstyle=\small,]


public Number m(Number x){
	return x;
}

\end{lstlisting}


























\newpage
\section{11-03-2019}
\textbf{WAITING FOR SOME DATA} \newline
Sembrerebbe che nessuno abbia preso appunti questo giorno \newline








\newpage
\section{18-03-2019}
\textbf{ECCEZIONI} \newline
Quando si lanciano eccezioni è bene ricordasi la differenza tra \textit{throw} e \textit{throws}
\begin{lstlisting}[basicstyle=\small,]
public void writeList() throws IOException {
	if(true)
		 throw new IOException("demo"); 
}
\end{lstlisting}
\textbullet\ throws viene messa affianco alla firma del metodo, seguita da una lista dei eccezioni, e serve per dichiarare quali eccezioni possono essere lanciate da un determinato metodo. \newline
\textbullet\ throw seguito da un oggetto di tipo eccezione, serve per lanciare l'eccezione. \newline
Si possono lanciare eccezione dei tipi \textit{Throwable}. Throwable è un super tipo di Exception e di tutte le classi lanciabili. \newline
Come regola generale quindi throw ha bisogno di essere seguito da un'espressione con tipo compatibile per poter essere lanciata.\newline
Il catch è lo strumento di binding per il throw. \newline
Le eccezioni non ritornano  per forza al chiamante se ritornano a chi se le prende. \newline
Il sistema try-catch è stato ideato per evitare delle forti anomalie del programma. \newline
Nel canale ufficiale del return vengo solamente ritornati i risultati "giusti", in caso contrario verrà lanciata un'eccezione: questo è lo stile richiesto per i linguaggi evoluti. Invece di complicare il tipo di ritorno usiamo le eccezioni. \newline
Al posto delle eccezioni possiamo definire  un tipo di ritorno che codifica il fatto che hai trovato o meno quello che cercavi. Questa tecnica non è molto leggibile per chi non ha scritto il codice, sarebbe meglio usare il design pattern " tipo-eccezione". Esso è utile anche perchè in questo modo non si può scrivere codice che non funzioni, mentre definendo un nuovo tipo è possibile.

\newpage









\newpage
\section{21-03-2019: COLLECTION, LIST E SET}
\begin{center}
\includegraphics[width=%
0.6\textwidth]{java-collection-hierarchy}
\end{center} 

\begin{center}
\includegraphics[width=%
0.6\textwidth]{MapInterface}
\end{center} 
\textbullet\ -> ITERABLE:Posso solo scorrere \newline
\textbullet\ ---> COLLECTION: Posso scorrere, aggiungere e togliere \newline
\textbullet\ -----> LIST: Posso scorrere e aggiungere o togliere con un indice \newline
\textbullet\ -------> ARRAYLIST:è sottotipo di list, in quanto implementa list! \newline
\digraph{ao}{rankdir=LR;

   a [label="Iterable" shape = "record"]; 
   b [label="Collection" shape = "record"]; 
   c [label="Set" shape = "record"]; 
   d [label="Hash" shape = "record"]; 
   e [label="List" shape = "record"]; 
   f [label="ArrayList" shape = "record"];    
   b->a; e->b; f->e; d->c; c->b} \newline

\noindent Per evitare di riprodurre codice si usano le classi astratte dalle quali poi si erediterà. 

\noindent Relazione di ordinamento: operatore binario che permette di mappare elementi di due insiemi diversi.

\noindent Una classe con metodi tutti statici non si può costruire. Rappresenta dunque un contenitore di metodi (è un pezzo di libreria).

\noindent \textbf{CARATTERISTICHE DI SET}\newline
\textbullet\ Gli elementi non sono duplicati \newline
\textbullet\ Gli elementi non sono ordinati in base a come sono inseriti \newline	
\textbullet\ Non vengono inseriti metodi nuovi, eredita solo quelli del padre \newline
\textbullet\ I metodi nuovi vengono messi nella classe che implementa l'interfaccia \newline
\textbullet\ Tiene un ordinamento che cerca di massimizzare le operazioni di ricerca ed estrazione \newline

\noindent \textbf{CARATTERISTICHE DI LIST}\newline
\textbullet\ Gli elementi possono essere duplicati \newline
\textbullet\ Gli elementi sono ordinati in base all'ordine di inserimento \newline	
\textbullet\ Rispetto ai metodi del padre ne aggiunge altri che permettono di leggere ed inserire elementi in un determinato indice \newline

\noindent \textbf{CODICE DI ESEMPIO}\newline
Il professore questo giorno ha caricato su github del codice chiamato: TinyJDK. Lo riporto per completezza:
\begin{lstlisting}[basicstyle=\small,]
/* Classe: MyIterable.java */
public interface MyIterable<E> {

    MyIterator<E> iterator();
    int find(E x) throws Exception;

}
\end{lstlisting}

\begin{lstlisting}[basicstyle=\small,]
/* Classe: MyIterator.java */
public interface MyIterator<E> {
    boolean hasNext();
    E next();
}
\end{lstlisting}

\begin{lstlisting}[basicstyle=\small,]
/* Classe: MyCollection.java */
import java.util.Collection;
import java.util.function.Function;

public interface MyCollection<T> extends MyIterable<T> {
    void add(T x);
    void clear();
    void remove(T x);   // da decidere se ci piace o no
    boolean contains(T x);
    boolean contains(Function<T, Boolean> p);
    int size();


}
\end{lstlisting}

\begin{lstlisting}[basicstyle=\small,]
/* Classe: MyList.java */

public interface MyList<T> extends MyCollection<T> {
    void add(int i, T x);
    T get(int i);
    void set(int i, T x);
}
\end{lstlisting}

\begin{lstlisting}[basicstyle=\small,]
/* Classe: MySet.java */
public interface MySet<T> extends MyCollection<T> {
}
\end{lstlisting}

\noindent \textbf{ESEMPIO IMPLEMENTAZIONE LIST}\newline
Vediamo un esempio di implementazione della classe List, collection in cui gli elementi vengono memorizzati in ordine di arrivo.

\begin{lstlisting}[basicstyle=\small,]
/* Classe: MyArrayList.java */
import java.util.Collection;
import java.util.function.Function;

public class MyArrayList<T> implements MyList<T> {

    private Object[] a;
    private int actualSize;

    public static class MyException extends Exception {
        public MyException(String s) {
            super(s);
        }
    }

/*    T[] toArray() {
        return (T[]) a;
    }*/

/*    public static Exception returnNow() {
        return new Exception("msg");
    }

    public static void throwNow() throws Exception {
        throw new Exception("msg");
    }

    public static void caller() throws Exception {
        Exception e = returnNow();
        throwNow();
    }

    public static void caller2() {
        try {
            caller();
        }
        catch (Exception e2) {
            // fai qualcosa con e2
        }

    }

    public void m(int x) throws Exception {
        MyException e = new MyException("error messagge");
        if (x < 0) throw e;
    }
  */

    public MyArrayList() {
        clear();
    }

    public static class NotFoundException extends Exception {
    }

	/* ritorna l'indice in cui è memorizzato un oggetto equivalente a 
	 * quello passato come parametro di ingresso */
    @Override
    public int find(T x) throws NotFoundException {
        int cnt = 0;
        MyIterator<T> it = iterator();
        while (it.hasNext())
        {
            T y = it.next();
            if (x.equals(y)) return cnt;
            ++cnt;
        }
        throw new NotFoundException();
    }




    public static void main3() {
        MyArrayList<Integer> c = new MyArrayList<>();
        try {
            int index = c.find(6);
            System.out.println("found at index = " + index);
        } catch (NotFoundException e) {
            try {
                int index = c.find(7);
            } catch (NotFoundException e1) {

            }

        }
    }

    @Override
    public boolean contains(T x) {
        for (int i = 0; i < actualSize; ++i) {
            Object o = a[i];
            if (o.equals(x)) return true;
        }
        return false;
    }


    @Override
    public boolean contains(Function<T, Boolean> p) {
        return false;
    }

    @Override
    public int size() {
        return actualSize;
    }


    @Override
    public void clear() {
        a = new Object[100];
        actualSize = 0;
    }

    @Override
    public void add(T o) {
        a[actualSize++] = o;
        if (actualSize >= a.length) {
            Object[] u = new Object[a.length * 2];
            for (int j = 0; j < a.length; ++j)
                u[j] = a[j];
            a = u;
        }
    }

    @Override
    public MyIterator<T> iterator() {
        return new MyIterator<T>() {
            private int pos = 0;

            @Override
            public boolean hasNext() {
                return pos <= actualSize;
            }

            @Override
            public T next() {
                return (T) MyArrayList.this.a[pos++];
            }
        };
    }

    @Override
    public void add(int i, T x) {

    }

    @Override
    public T get(int i) {
        return (T) a[i];
    }

    @Override
    public void set(int i, T x) {
        a[i] = x;
    }

    @Override
    public void remove(T x) {

    }

}
\end{lstlisting}

\noindent \textbf{ESEMPIO IMPLEMENTAZIONE SET}\newline
Vediamo un esempio di implementazione della classe Set, collection in cui gli elementi vengono inseriti secondo una certa logica interna. In questo caso gli elementi verranno tenuti secondo l'ordine imposto o da comparator o da comparable.

\begin{lstlisting}[basicstyle=\small,]
/* Classe: MyArrayListSet.java */
import java.util.ArrayList;
import java.util.Arrays;
import java.util.Collections;
import java.util.Comparator;
import java.util.function.Function;

public class MyArrayListSet<T extends Comparable<T>> implements MySet<T> {
    private final Comparator<T> p;
    private final ArrayList<T> a;

    public MyArrayListSet(Comparator<T> p) {
        this.a = new ArrayList<T>();
        this.p = p;
    }

    @Override
    public void add(T x) {
        if (!contains(x)) {
            a.add(x);
            sort();
        }
    }

    private void sort() {
        Collections.sort(a, p);
    }

    @Override
    public void clear() {
        a.clear();
    }

    @Override
    public void remove(T x) {
        a.remove(x);
    }

    @Override
    public boolean contains(T x) {
        return a.contains(x);
    }

    @Override
    public boolean contains(Function<T, Boolean> p) {
        return a.contains(p);
    }

    @Override
    public int size() {
        return a.size();
    }

    @Override
    public MyIterator<T> iterator() {
        return a.iterator();
    }

    @Override
    public int find(T x) throws Exception {
        return a.find(x);
    }
}
\end{lstlisting}








\newpage
\section{25-03-2019: COMPARABLE, COMPARATOR, LIST E SET}
\noindent Ha continuato quello che ha fatto la scorsa lezione... in più ha spiegato le interfacce \textit{comparable} e \textit{comparator} \newline 

\noindent \textbf{COMPARABLE E COMPARATOR}\newline
\textbullet\ \textit{Comparable}: è un'interfaccia con un unico metodo compareTo(T obj) che compara se stesso con un oggetto di tipo T. E' quindi interfaccia funzionale. \newline
\textbullet\ \textit{Comparator}: è un interfaccia con molti metodi, il metodo compare(T a, T b) ci permette di realizzare una classe che compara oggetti tra di loro. \newline
Vediamo un esempio di utilizzo preso dalla classe Sort: \newline
\begin{lstlisting}[basicstyle=\small,]
Sort(T a) /* Posso usare questo metodo quando gli oggetti sono 
		   * sicuramente di tipo comparable. In fase di ordinamento
		   * richiamerò il compare su ogni elemento */
Sort(T a, Comparator<T>) /* Uso questo metodo quando voglio rendere
		     * gli oggetti comparabili al momento, senza usare
		     * l'interfaccia comparable  */
		     
\end{lstlisting}

\noindent \textbf{RIPRENDO IL CODICE VISTO L'ALTRA VOLTA + ESEMPIO COMPARATOR}\newline
L'altra volta ho realizzato direttamente la classe \textit{MyArrayListSet.java} implementando \textit{MySet}.
Un modo alternativo di farlo consiste nel creare una classe astratta con dei metodi di default ed estenderla. 

\begin{lstlisting}[basicstyle=\small,]
/* Classe: MyAbstractArrayListSet.java */
import java.util.*;
import java.util.function.Function;

public abstract class MyAbstractArrayListSet<T> implements MySet<T> {
    protected final ArrayList<T> a;

 	/* un costruttore protetto significa che non posso istanziare
 	 * la classe stessa nel codice di un fantomatico main, ma 
 	 * che sono costretto ad ereditarla */
    protected MyAbstractArrayListSet() {
        this.a = new ArrayList<T>();
    }

    @Override
    public void add(T x) {
        if (!contains(x)) {
            a.add(x);
            sort();
        }
    }

    protected abstract void sort();

    @Override
    public void clear() {
        a.clear();
    }

    @Override
    public void remove(T x) {
        a.remove(x);
    }

    @Override
    public boolean contains(T x) {
        return a.contains(x);
    }

    @Override
    public boolean contains(Function<T, Boolean> p) {
        return a.contains(p);
    }

    @Override
    public int size() {
        return a.size();
    }

    @Override
    public MyIterator<T> iterator() {
        Iterator<T> it = a.iterator();
        return new MyIterator<T>() {

            @Override
            public boolean hasNext() {
                return it.hasNext();
            }

            @Override
            public T next() {
                return it.next();
            }
        };
    }

    @Override
    public int find(T x) throws Exception {
        int r = a.indexOf(x);
        if (r < 0) throw new Exception("not found");
        return r;
    }
}

\end{lstlisting}



\begin{lstlisting}[basicstyle=\small,]
/* Classe: MyArrayListSet.java */
/* è stata modificata la classe dell'altra volta*/

import java.util.*;
import java.util.function.Function;

public class MyArrayListSet<T> extends MyAbstractArrayListSet<T> {

    private final Comparator<T> p;

    public MyArrayListSet(Comparator<T> p) {
        super();
        this.p = p;
    }

    @Override
    protected void sort() {
        Collections.sort(a, p);
    }

}

\end{lstlisting}

\noindent \textbf{ESEMPIO COMPARABLE}
\begin{lstlisting}[basicstyle=\small,]
/* Classe: MySortedSet.java */
public interface MySortedSet<T extends Comparable<T>> extends MySet<T> {

}
\end{lstlisting}

\begin{lstlisting}[basicstyle=\small,]
/* Classe: MyArrayListSortedSet.java */
import java.util.*;
import java.util.function.Function;

public class MyArrayListSortedSet<T extends Comparable<T>>
        extends MyAbstractArrayListSet<T>
        implements MySortedSet<T> {
        /* implements è stato aggiunto la lezione
         * successiva */

    public MyArrayListSortedSet() {
        super();
    }

    @Override
    protected void sort() {
        Collections.sort(a);
    }

}
\end{lstlisting}

\begin{lstlisting}[basicstyle=\small,]
/* Classe: Main.java */
import java.util.*;

public class Main {

    public static class Plant {
        private int height;

    }

    public static class Animal implements Comparable<Animal> {
        private int weight;
        private String name;

        public Animal(String name, int w) {
            this.name = name;
            this.weight = w;
        }

        public int getWeight() { return weight; }

        public String getName() {
            return name;
        }

        @Override
        public int compareTo(Animal o) {
            return this.weight - o.weight;
            /*if (this.weight == o.weight) return 0;
            else if (this.weight > o.weight) return 1;
            else return -1;*/
        }
    }

    public static class Dog extends Animal {

        public Dog(String name, int w) {
            super(name, w);
        }
    }

    public static void main(String[] args) {

        List<Animal> a = new ArrayList<>();
        Collections.sort(a);

        List<Plant> b = new ArrayList<>();
        Collections.sort(b, new Comparator<Plant>() {
            @Override
            public int compare(Plant x, Plant y) {
                return x.height - y.height;
            }
        });

    }
}
\end{lstlisting}


\newpage
\section{28-03-2019: MAP}
\noindent Affinchè due classi siano confrontabili devono implementare il metodo compareTo \newline
Solitamente si usa il modificatore protected invece che private in modo tale che da fuori non possano essere modificate, ma possano essere viste quando vengono ereditate. \newline

\noindent \textbf{MAPPE}\newline
Una mappa è un oggetto che mappa chiavi con valori. Non ci possono essere chiavi duplicate e ogni chiave può mappare al più un valore. \newline
Una mappa rappresenta una collection associativa. \newline
Una mappa è parametrica su due tipi di argomenti. Uno rappresenta il dominio (chiave), e l'altro rappresenta il codominio (valore). La gerarchia delle mappe è slegata dalla gerarchia delle collection. \newline
Per la documentazione java sulle mappe consultare questo \href{https://docs.oracle.com/javase/8/docs/api/java/util/Map.html }{SITO} \newline
Una mappa è una collection associativa, con coppie associate a valori.

\noindent Il foreach di JAVA funziona solo con le librerie del JDK originale di JAVA.\newline
Super può essere usato per il costruttore o per metodi che eredito, non è di perse un metodo e non ha un tipo.

\noindent Durante la lezione il professore ha aggiunto nel repository di github il seguente codice: 


\begin{lstlisting}[basicstyle=\small,]
/* Classe: Pair.java */
public class Pair<A, B> {
    private A a;
    private B b;

    public Pair(A a, B b) {
        this.a = a;
        this.b = b;
    }

    public A getFirst() { return a; }
    public B getSecond() { return b; }

}
\end{lstlisting}

\begin{lstlisting}[basicstyle=\small,]
/* Classe: NotFoundException.java */
public class NotFoundException extends Exception {
}
\end{lstlisting}

\begin{lstlisting}[basicstyle=\small,]
/* Classe: MyMap.java */
public interface MyMap<K, V> extends MyCollection<Pair<K, V>> {
	/*K = sono le chiavi */
	/*V = sono i valori  */
    void put(K key, V value);
    V get(K key) throws NotFoundException;

}
\end{lstlisting}

\noindent \textbf{IMPLEMENTAZIONE DI MAPPE}\newline
\noindent Decidiamo di implementare la nostra mappa in due maniere diverse! La prima è la più facile che ci viene in mente:

\begin{lstlisting}[basicstyle=\small,]
/* Classe: MyListMap__old.java */
import java.util.function.Function;

public class MyListMap__old<K, V> implements MyMap<K, V> {

    private MyList<Pair<K, V>> l;

    public MyListMap__old() {
        this.l = new MyArrayList<>();
    }

    @Override
    public void put(K key, V value) {
        l.add(new Pair<K, V>(key, value));
    }

    @Override
    public V get(K key) throws NotFoundException {
        MyIterator<Pair<K, V>> it = l.iterator();
        while (it.hasNext()) {
            Pair<K, V> p = it.next();
            if (key.equals(p.getFirst())) return p.getSecond();
        }
        throw new NotFoundException();
    }

    @Override
    public void add(Pair<K, V> x) {
        put(x.getFirst(), x.getSecond());
    }

    @Override
    public void clear() {
        l.clear();
    }

    @Override
    public void remove(Pair<K, V> x) {
        l.remove(x);
    }

    @Override
    public boolean contains(Pair<K, V> x) {
        return l.contains(x);
    }

    @Override
    public boolean contains(Function<Pair<K, V>, Boolean> p) {
        return l.contains(p);
    }

    @Override
    public int size() {
        return l.size();
    }

    @Override
    public MyIterator<Pair<K, V>> iterator() {
        return l.iterator();
    }

    @Override
    public int find(Pair<K, V> x) throws Exception {
        return l.find(x);
    }
}
\end{lstlisting}

\noindent La sconda implementazione è la più coincisa:

\begin{lstlisting}[basicstyle=\small,]
/* Classe: MyListMap.java */
public class MyListMap<K, V> extends MyArrayList<Pair<K, V>>
        implements MyMap<K, V> {

    @Override
    public void put(K key, V value) {
        add(new Pair<K, V>(key, value));
    }

    @Override
    public V get(K key) throws NotFoundException {
        MyIterator<Pair<K, V>> it = iterator();
        while (it.hasNext()) {
            Pair<K, V> p = it.next();
            if (key.equals(p.getFirst())) return p.getSecond();
        }
        throw new NotFoundException();
    }
}
\end{lstlisting}

\begin{lstlisting}[basicstyle=\small,]
/* Classe: Main.java */
/* è stata modificata la classe dell'altra volta*/
/* riporto solo la modifica: ha aggiunto main2 */

    public static void main2() {
        MyCollection<Pair<String, Integer>> rubrica = new MyListMap<>();
        rubrica.add(new Pair<>("Alvise", 34712345));
        rubrica.add(new Pair<>("Diego", 987654321));
    }



\end{lstlisting}

\noindent Finita la lezione il professore ha deprecato la classe: MyListMap\_ \_ old.java
\end{document}
