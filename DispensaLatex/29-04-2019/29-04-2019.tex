

\newpage
\section{29-04-2019: PRODUTTORE CONSUMATORE}

\noindent \textbf{DESIGN PATTNER: PRODUTTORE CONSUMATORE} \newline
La filosofia è questa: qualcuno produce le cose e qualcun altro le utilizza. Ad esempio un thread produce i dati mettendoli in una coda ed un'altro thread li consuma estraendoli. \newline
Rilassare un'eccezione significa salire di livello nella gerarchia. Invece che raccogliere un'eccezione specifica le raccolgo tutte, ad esempio usando: catch(Exception e). \newline
Le stringhe in java non sono mutabili. \newline
Tutte le strutture che implementano cose bloccanti sono anche thread safe. \newline
Un oggetto non viene preso in carico dal garbage collector quando è dentro una variabile e quando è dentro ad una struttura dati. Questo, se non si presta attenzione, può portare al fenomeno della memory leak (ingrossamento della memoria a dismisura). \newline

\noindent \textbf{CODE BLOCCANTI} \newline
Per trattare il tema produttore consumatore andremo ad utilizzare delle code bloccanti. Questo perchè abbiamo due problemi: il primo è che non possiamo aggiungere dati alla coda se essa è piena, il secondo è che non possiamo retribuire dati da una coda se essa è vuota. Le code bloccanti risolvono questi problemi esportando le seguenti api: \newline
\textbullet\ put(): inserisce un elemento nella coda aspettando che ci sia uno slot libero. \newline
\textbullet\ add(): inserisce un elemento nella coda e ritorna true, se non riesce ad inserirlo (coda piena) ritorna false. \newline
\textbullet\ take(): prende e rimuove un elemento dalla testa della coda aspettando, in maniera bloccante, che la coda non sia vuota. \newline
\textbullet\ poll(timeout): prende un elemento dalla testa della coda aspettando, per un certo timeout.\newline

\noindent \textbf{CODE BLOCCANTI VS NON } \newline
Abbiamo già visto un esempio di coda non bloccante: ConcurrentLinkedQueue \newline
La differenza tra le due è che una blocca le richieste di inserimento ed accesso ai dati, aspettando che le condizioni necessarie vengano soddisfatte, mentre l'altra ritorna false nel caso non riuscisse a svolgere subito le sue operazioni.\newline

\noindent \textbf{CLASSE RANDOM } \newline
La classe Random ci aiuta a generare numeri casuali in Java. Essa possiede due costruttori, uno vuoto ed uno che prende come parametro un tipo intero long. Visto che le i generatori di numeri casuali non producono effettivamente numeri veramente casuali impostare un certo seed ci permette di settare lo stato iniziale del generatore di numeri casuali. Dati due generatori Random con lo stesso stato iniziale (o quindi lo stesso seed) otterremo in uscita una sequenza di numeri identica.
\begin{lstlisting}
/* Sceglie un seed casuale */
public Random();
/* Imposto io un seed */
public Random(long seed);
\end{lstlisting}
\noindent tra i metodi di questa classe troviamo :
\begin{lstlisting}
/* Ritorna un numero della sequenza pseudo
 * casuale che può variare da 0 al massimo
 * numero rappresentabile da un int */
public int nextInt();
/* Ritorna un numero della sequenza pseudo
 * casuale che può variare da 0 ad un massimo
 * di bound -1 */
public int nextInt(int bound);
\end{lstlisting}


\noindent \textbf{CODICE DEL PROFESSORE} \newline
\begin{lstlisting}
/* Classe: Main.java */

package patterns.consumerproducer;

import java.util.Random;
import java.util.concurrent.BlockingQueue;
import java.util.concurrent.LinkedBlockingQueue;

public class Main {
    private static BlockingQueue<String> q = new LinkedBlockingQueue<>();
    private static Random rand = new Random();

    public static class Consumer extends Thread{
        @Override
        public void run(){
            while(true)
                try{
                    Thread.sleep(rand.nextInt(500));

                    String s = q.take();
                    System.out.println("Consumer: "+s);

                }catch (InterruptedException e){
                    e.printStackTrace();
                }
        }
    }

    public static class Producer extends Thread{

        @Override
        public void run(){

               /* ad ogni ciclio costruiamo un nuovo generatore
                * di numeri casuali, ma non viene cambiato il 
                * seme facciamo anche heap pllution (riempiamo
                * lo heap di oggetti non referenziati) */
               while(true){
                   try {
                       Thread.sleep(rand.nextInt(500));
                       /* Random rand= new Random(); */
                       int len= rand.nextInt(10); 
                       /* produce un numero tra 0 e 100 */
                       String s= ""; 
                       /* rappresenta uno zucchero sintattico
                        * per indicare una stringa statica
                        * all'interno della classe.
                        * la stringa vuota è un oggetto
                        * già esistente. è immutabile */
                       for(int i=0;i<len;i++) {
                           /*
                           s=s+String.format("%d",i); //usa l'opearndo string string
                           s=s+i; //usa loprebado + con string int
                           s=s+(Integer) i; //usa l'operando con string Integer
                           s=s+((Integer)i).toString();
                            */
                           s = s + String.format("%d", i);
                           /* ogni volta ne viene creata
                            * una nuova.
                            * s=s+i; */

                           /* le variabili ospitano oggetti */
                           

                           /* s computa il contenuto della striga
                            * String.format computa una nuova stinga
                            */
                       }
                       System.out.println("Producer: "+s);
                       q.add(s);
                   }catch (Exception e){
                       e.printStackTrace();
                   }

               }
        }

    }
    public static void main(String[] args){
        new Producer().start();
        new Consumer().start();
    }
}
\end{lstlisting}