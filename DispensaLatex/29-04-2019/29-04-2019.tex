

\newpage
\section{29-04-2019: PRODUTTORE CONSUMATORE}

\noindent \textbf{DESIGN PATTNER: PRODUTTORE CONSUMATORE} \newline
La filosofia è questa: qualcuno produce le cose e qualcun altro le utilizza. Ad esempio un thread produce i dati mettendoli in una coda ed un'altro thread li consuma estraendoli. \newline
Rilassare un'eccezione significa salire di livello nella gerarchia. Invece che raccogliere un'eccezione specifica le raccolgo tutte, ad esempio usando: catch(Exception e). \newline
Le stringhe in java non sono mutabili. \newline
Tutte le strutture che implementano cose bloccanti sono anche thread safe. \newline
Un oggetto non viene preso in carico dal garbage collector quando è dentro una variabile e quando è dentro ad una struttura dati. Questo, se non si presta attenzione, può portare al fenomeno della memory leak (ingrossamento della memoria a dismisura). \newline

\noindent \textbf{CODICE DEL PROFESSORE} \newline
 


