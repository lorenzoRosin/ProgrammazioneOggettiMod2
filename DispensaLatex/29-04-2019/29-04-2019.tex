

\newpage
\section{29-04-2019: PRODUTTORE CONSUMATORE}

\noindent \textbf{DESIGN PATTNER: PRODUTTORE CONSUMATORE} \newline
La filosofia è questa: qualcuno produce le cose e qualcun altro le utilizza. Ad esempio un thread produce i dati mettendoli in una coda ed un'altro thread li consuma estraendoli. \newline
Rilassare un'eccezione significa salire di livello nella gerarchia. Invece che raccogliere un'eccezione specifica le raccolgo tutte, ad esempio usando: catch(Exception e). \newline
Le stringhe in java non sono mutabili. \newline
Tutte le strutture che implementano cose bloccanti sono anche thread safe. \newline
Un oggetto non viene preso in carico dal garbage collector quando è dentro una variabile e quando è dentro ad una struttura dati. Questo, se non si presta attenzione, può portare al fenomeno della memory leak (ingrossamento della memoria a dismisura). \newline

\noindent \textbf{CODICE DEL PROFESSORE} \newline
 

\begin{lstlisting}[basicstyle=\small,]
package patterns.consumerproducer;

import java.util.Random;
import java.util.concurrent.BlockingQueue;
import java.util.concurrent.LinkedBlockingQueue;

public class Main {
    private static BlockingQueue<String> q = new LinkedBlockingQueue<>();
    private static Random rand = new Random();

    public static class Consumer extends Thread{
        @Override
        public void run(){
            while(true)
                try{
                    Thread.sleep(rand.nextInt(500));

                    String s = q.take();
                    System.out.println("Consumer: "+s);

                }catch (InterruptedException e){
                    e.printStackTrace();
                }
        }
    }

    public static class Producer extends Thread{

        @Override
        public void run(){

               /* ad ogni ciclio costruiamo un nuovo generatore
                * di numeri casuali, ma non viene cambiato il 
                * seme facciamo anche heap pllution (riempiamo
                * lo heap di oggetti non referenziati) */
               while(true){
                   try {
                       Thread.sleep(rand.nextInt(500));
                       /* Random rand= new Random(); */
                       int len= rand.nextInt(10); 
                       /* produce un numero tra 0 e 100 */
                       String s= ""; 
                       /* rappresenta uno zucchero sintattico
                        * per indicare una stringa statica
                        * all'interno della classe.
                        * la stringa vuota è un oggetto
                        * già esistente. è immutabile */
                       for(int i=0;i<len;i++) {
                           /*
                           s=s+String.format("%d",i); //usa l'opearndo string string
                           s=s+i; //usa loprebado + con string int
                           s=s+(Integer) i; //usa l'operando con string Integer
                           s=s+((Integer)i).toString();
                            */
                           s = s + String.format("%d", i);
                           /* ogni volta ne viene creata
                            * una nuova.
                            * s=s+i; */

                           /* le variabili ospitano oggetti */
                           

                           /* s computa il contenuto della striga
                            * String.format computa una nuova stinga
                            */
                       }
                       System.out.println("Producer: "+s);
                       q.add(s);
                   }catch (Exception e){
                       e.printStackTrace();
                   }

               }
        }

    }
    public static void main(String[] args){
        new Producer().start();
        new Consumer().start();
    }
}
\end{lstlisting}