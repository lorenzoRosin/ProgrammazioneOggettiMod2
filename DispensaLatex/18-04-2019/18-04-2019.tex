

\newpage
\section{18-04-2019}
\noindent \textbf{CONTINUAZIONE DEI THREAD} \newline
Thread.currentThread() ritorna un oggetto di tipo thread che rappresenta se stesso. Ritorna una istanza della classe Thread in carica di eseguire il thread. \newline
Il metodo run non può lanciare nessuna eccezione di tipo checked \newline
Quando si usano i timeout si è soliti specificare quanto tempo e l'unità di misura. Essa si può specificare usando TimeUnit.UNITADIMISURA. \newline

\noindent \textbf{ESPRESSIONI LAMBDA} \newline
Sintassi: \newline
\textbullet\ $ (\tau_{1} x_{1} , .... , \tau_{n} x_{n}) -> E $ Dove E è un'espressione \newline
Vengono passati anche i tipi in quanto non c'è un type inference completo. Inoltre l'espressione della lambda può essere anche un blocco. \newline
Vediamo degli esempi: \newline
\textbullet\ $ (int n) -> (n+1) $ E' corretta \newline
\textbullet\ $ (int x) -> return x+1 $ E' errata. infatti il return non è un'espressione, è uno statament, dunque ci vanno le graffe. \newline
\textbullet\ $ (int x) -> {return x+1} $ E' corretta \newline
\textbullet\ $ () -> 1 $ E' corretta \newline
In generale se l'espressione lambda è semplice si possono omettere le graffe, mentre se ha uno statament bisogna metterci le graffe. \newline


\noindent \textbf{CODICE DEL PROFESSORE} \newline

 


