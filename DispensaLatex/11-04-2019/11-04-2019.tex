

\newpage
\section{11-04-2019: THREAD POOL}
\noindent \textbf{CONCURRENTLINKEDQUEUE} \newline
Aggiunto appunti miei... ConcurrentLinkedQueue è semplicemente una coda di tipo FIFO, che può essere usata da più thread contemporaneamente.Essa ci garantisce che se un thread aggiunge un elemento mentre un'altro thread ne sta aggiungendo un'altro non si verifichino problemi di memoria. Nulla vieta però che se la coda ha un elemento e cerchiamo di leggerlo, un'altro thread non l'abbia già letto e ci ritroviamo a leggere da una coda vuota. La lettura di una coda vuota ritorna null.

\noindent \textbf{wait() and notify()} \newline
\textbullet\ \textit{wait():} Forza il thread corrente ad aspettare. Il thread viene sospeso fino a che qualcuno non chiama notify \newline
\textbullet\ \textit{notify():} Risveglia e mette in run i thread che erano stati fermati con wait.


\noindent Gli appunti dell'altra volta sono gli stessi di questa lezione, il professore ha aggiunto un po di codice.
 


\begin{lstlisting}[basicstyle=\small,]
/* Classe: SynchronizedMain.java */
/* ha modificato il codice della scorsa volta */
/* riporto solo la modifica */
    public static void main(String[] args) {
//        MyThread th = new MyThread(500);
//        th.start();
        new MyThread(500).start();

        new Thread(() -> loop(400)).start();

        loop(300);
    }
\end{lstlisting}

\noindent Sucecssivamente è tornato a concentrarsi sui thread pool.

\begin{lstlisting}[basicstyle=\small,]
/* Classe: ThreadPool.java */
package threads;

public class ThreadPool implements Queue<Thread> {



}

\end{lstlisting}

\begin{lstlisting}[basicstyle=\small,]
/* Classe: ThreadPool.java */
/* Questa classe non è ancora completa 
 * é una bozza, verrà completata nelle prossime
 * lezioni */
import java.util.Queue;
import java.util.concurrent.ConcurrentLinkedQueue;

public class ThreadPool {
    private Queue<PooledThread> q = new ConcurrentLinkedQueue<>();

    public static class PooledThread extends Thread {
        private Runnable r;
    }

    public PooledThread acquire(Runnable cb) {
        if (q.isEmpty()) {
            return new PooledThread();
        }
        else {
            PooledThread p = q.poll();
            p.notify();
        }
    }

    public void release(PooledThread t) {
        q.add(t);
    }

    public static void threadMain(PooledThread p) {
        while (true) {
            try {
                p.wait();
                p.r.run();  // TODO: potrebbe essere null
            } catch (InterruptedException e) {
                e.printStackTrace();
            }
        }
    }

    public static void main(String[] args) {
        ThreadPool pool = new ThreadPool();
        Thread t1 = pool.acquire();
        t1.
    }
}


\end{lstlisting}




\noindent Sucecssivamente ha modificato ancora la synconized main

\begin{lstlisting}[basicstyle=\small,]
/* Classe: SynchronizedMain.java */
/* è stata modificata la classe di prima    */
/* ne riporto solamente la modifica al main */
    public static void main(String[] args) {
        MyThread th1 = new MyThread(500);
        th1.start();    // spawn thread 1

        // non serve conservare l'oggetto thread in una variabile se non è necessario
        new Thread(() -> loop(400)).start();    // spawn thread 2 con un runnable in costruzione

        // esegue lo stesso codice anche col main thread
        loop(300);  
    }

\end{lstlisting}










