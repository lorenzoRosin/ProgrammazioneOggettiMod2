

\newpage
\section{21-02-2019}
\par

\textit{REFLECTION} è una tecnica per conoscere i tipi e il contenuto delle classi  a runtime. \newline
Se voglio conoscere il tipo dell'enclosing class (classe che contiene) posso fare : nome.classe.this.nome \newline

\textit{BINDING} avviene anche con i tipi \newline
I parametri di una funzione sono binding nello scope della funzione.\newline
I type argument fanno binding con i type parameter, esattamente come avviene per le funzioni tra argomenti e parametri. \newline

Quando si programma con i generics si PROPAGANO. \newline

TYPE ERASURE: cancellazione dei tipi: java lo fa quando compila butta via i generics generando classi non anonime e li sostituisce con Object: il motivo è per mantenere la compatiblità con il vecchio codice che non aveva generics. Quindi i generics sono verificati dal compilatore e poi cancellati per eseguire.














