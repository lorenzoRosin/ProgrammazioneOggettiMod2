

\newpage
\section{28-02-2019: COVARIANZA e CONTROVARIANZA}
\textbf{COVARIANZA e CONTROVARIANZA DEI TIPI }\newline
Scrivendo  $C \leq A$ intendiamo che C è sottotipo di A. Definito l'operatore minore uguale passiamo alle definizioni vere e propie.

\noindent \textbf{VARIANZA} \newline
$C_{1} <\tau_{1}> \leq C_{2}<\tau_{2}> \Leftrightarrow C_{1} \leq C_{2} \bigwedge \tau_{1}\equiv \tau_{2} $
\newline
Questa regola del type system di \textit{java} ci dice che il linguaggio non è COVARIANTE, in quanto i generics non cambiano. 

\noindent \textbullet\ Esempio: ArrayList<cane> $\leq$ List<cane> (sottotipo) \newline
\textbullet\ Esempio: ArrayList<cane> $\nleq$ List<Animali> \newline
L'ultima formula non è covariante, se fosse possibile si avrebbe una doppia subsunzione sul guscio interno e sul tipo esterno \newline

\noindent \textbf{CONTROVARIANZA} \newline
Quando eredito un metodo di una classe posso sovrascriverlo usando l'overriding. Quando lo faccio un linguaggio di programmazione si dice controvariante se mi è possibile far scendere (specializzare) il tipo di ritorno, e al contempo mi è possibile far salire (rendere più generico) l'argomento di ingresso. Vediamo un esempio (\textit{NON VALIDO IN JAVA}):
\begin{lstlisting}
/* Data questa gerarchia di classi: */
  public class Cane extends Animale{
	/* nella classe Cane faccio l'override
 	 * del metodo ereditato dalla classe
 	 * Animale */  
	public Cane m(Cane c){
		return c;
	}
  }
  
 public class PastoreTedesco Extends Cane{
 	@Override
	public PastoreTedesco m(Animale c){
		return new PastoreTedesco;
	}
	/* rispetto al metodo originale
	 * il tipo di ritorno del metodo 
	 *		(PastoreTedesco) scende (sottoclassi)
 	 * il tipo del parametro di ingresso 
 	 * 		(Animale) sale (superclasse) */	
 }
\end{lstlisting}

\noindent Detto questo è bene specificare che Java supporta i solo i tipi di ritorno controvarianti, è possibile controvariare SOLO il tipo di ritorno del metodo, ed è possibile farlo solamente scendendo (sottotipo). Ed è possibile Fare questo nella fase di overriding, non in quella di overloading. In caso di overloading il tipo di ritorno deve essere lo stesso.

\noindent Il seguente è l'esempio corretto in java:
\begin{lstlisting}
public Cane m (Cane c){return c;}

@Override
public PastoreTedesco m(Cane c){return new PastoreTedesco();}
\end{lstlisting}

\noindent \textbf{SOUNDNESS IN JAVA} \newline
\textbullet\ SOUND : un programma che compila può essere eseguito senza errori \newline
\textbullet\ SOUND JAVA: un programma compila e termina, a meno di una eccezione. \newline
Un esempio di SOUND JAVA è il seguente:in java è possibile avvenga un segmentation fault non per un problema di casting, ma solamente se accediamo ad un indice di un array che non abbiamo allocato, quindi il programma termina a meno di una eccezione. Ci sono invece linguaggi dove non esistono gli array, quindi non accadrà mai segmentation fault e il codice terminerà sempre alla fine senza eccezioni, ovviamente senza fare i controlli di semantica. 

\noindent \textbf{WILDCARDS} \newline
\noindent Recentemente è stato inserito un pattern che permette anche in Java la covarianza: sono i generics con i wildcards. 
\begin{lstlisting}
Arraylist<? extends Animale> m = new List<Cane>();
\end{lstlisting}
\noindent Da questo si capisce che la covarianza può essere usata, ma solamente se esplicitata con il wildcard. \newline
Sono molto usati perchè i wildcard non sono tipi del primo ordine, infatti il \textit{tipo} "? extends classe" non esiste!!!Non posso definire una variabile come segue: 
\begin{lstlisting}
? extends Animale m = new Cane();
/* questa sintassi si può usare solo come type argument */
\end{lstlisting}
Significato: permettono la covarianza, sono tipi temporanei che non possono essere scritti nel codice, però possono essere subsunti con il get(). \newline
Un altro DESIGN PATTERN: callback














