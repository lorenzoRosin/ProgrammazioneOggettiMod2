

\newpage
\section{28-02-2019}
\par

\textit{COVARIANZA e CONTROVARIANZA} dei tipi \newline
\textit{VARIANZA} : \newline
$C_{1} <\tau_{1}> \leq C_{2}<\tau_{2}> \Leftrightarrow C_{1} \leq C_{2} \bigwedge \tau_{1}\equiv \tau_{2} $
Questa regola del type system di java si dice che il linguaggio NON è COVARIANTE, in quanto i generics non cambiano. \newline

Esempio: ArrayList<cane> è minore uguale a List<cane> (sottotipo) \newline
Esempio: ArrayList<cane> NON è minore uguale a List<Animali> \newline
L'ultima formula non è covariante, se fosse possibile si avrebbe una doppia soluzione sul guscio interno e il tipo esterno \newline

\textit{CONTROVARIANZA} : \newline
Quando eredito posso fare l'overide, quando lo faccio il tipo di ritorno è constrovariante (uno sale e uno scende. Il parametro sale ( si specializza) e il ritorno scende (si despecializza). \newline





\begin{lstlisting}[basicstyle=\small,]

\\nella classe Animale:
public Cane m(Cane c){
	return c;
}

\\nella classe Cane:
@Override
public PastoreTedesco m(Animale c){ return new PastoreTedesco;}
\\ il tipo di ritorno del metodo (PastoreTedesco) scende (sottoclassi)
\\ il tipo del parametro di ingresso (Animale) sale (superclasse)

\end{lstlisting}

In java è possibile controvariare solo il tipo di ritorno del metodo, solo scendendo (sottotipo) \newline


\begin{lstlisting}[basicstyle=\small,]

public Cane m (Cane c){return c;}

@Override
public PastoreTedesco m(Cane c){return new PastoreTedesco();}

\end{lstlisting}

SOUND : un programma che compila può essere eseguito \newline
SOUND JAVA: un programma che compila e termina, a meno di una eccezione. \newline
In java è possibile avvenga un segmentation fault non per un problema di casting, ma solamente se accediamo ad un indice di un array non abbiamo allocato. \newline




Ci sono linguaggi dove non esistono gli array, quindi non accadrà mai segmentation fault e il codice terminerà sempre, ovviamente senza fare i controlli di semantica. \newline
Recentemente è stato inserito un pattern che qualcosa la covarianza: \newline


\begin{lstlisting}[basicstyle=\small,]

Arraylisti<? extends Animale> m = new Arraylist<Cane>();

\end{lstlisting}

Da questo si capisce che la covarianza può essere usata, ma solamente se esplicitata con il wildcard. \newline
Sono molto usati perchè non sono tipi del primo ordine \newline
Non posso definire una variabile: \newline

\begin{lstlisting}[basicstyle=\small,]
? extends Animale m = new Cane();
\\ questa sintassi si puo usare solo come type argument
\end{lstlisting}

Significato: permettono la covarianza, sono tipi temporanei che non possono essere scritti nel codice, però possono essere sostituiti con il get(). \newline
Un altro DESIGN PATTERN: callback














