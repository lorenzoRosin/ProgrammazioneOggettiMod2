
 
\newpage
\section{15-02-2019}
\textbf{INTERFACCE}\newline
\begin{lstlisting}[basicstyle=\small,]

	public interface Iterator<T>{}

\end{lstlisting}
L'interfaccia è un contratto, nel senso che mette a disposizione una serie di metodi che ogni classe che estende quell'interfaccia deve obbligatoriamente implementare, pena un errore durante la fase di compilazione.In java quindi si può scrivere del codice ancora prima di sapere come si potrebbe implementare.

\noindent Facciamo un esempio: il "contratto" di iteratore è il seguente: \newline
\textbullet\ boolean hasNext(); \newline
\textbullet\ T next(); \newline
\textbullet\ void remove()\newline
Data una certa classe che può non essere sotto al nostro controllo non abbiamo bisogna sapere necessariamente come sono stati implementati i suoi metodi, ma ci ti basta sapere che esistono per poter dire se sia o meno un iteratore. \newline
Esempio di definizione di un metodo con iteratore come input: 
\begin{lstlisting}[basicstyle=\small,]

	public statics void scorri(Iteratore <Integer> it){
		while(it.hasnext()){
			integer n = it.next();
		}
	}

\end{lstlisting}
Esempio di utilizzo 
\begin{lstlisting}[basicstyle=\small,]

	scorri(new Iterator<>(){
		...
		...
		...
	});

\end{lstlisting}
Quest'ultima è un'espressione, o come meglio dire, un'oggetto fatto al volo. Questa sintassi è stata creata appositamente per le interfacce (dato che non si possono istanziare direttamente), senza dover andare a definire una classe con la classica implementazione dell'interfaccia. 

\noindent \textit{ANONYMOUS CLASS} meccanismo comodo per design pattern come le call-back. 

\noindent Questa implementazione garantisce che la funzione sia \textit{SOUND}, e non crasherà mai a \textit{RunTime}

\noindent \textbf{IMPLEMENTARE INTERFACCE}\newline
1) Con implements: \newline
\textbullet\ controlla i metodi che hai implementato all'interno della classe \newline
\textbullet\ assicura che siano implementati tutti 

\noindent Tipi delle interfacce \newline
Iterator $\Rightarrow$ non è un tipo \newline
Iterator<T> $\Rightarrow$ è un tipo 

\noindent \textbf{NOTAZIONE BNS} \newline
BNS è il nome della notazione e serve per poter dare delle regole grammaticali. E' una notazione che definisce la sintassi delle espressioni\newline
Iterator da solo, sintatticamente, sarebbe un tipo. Ma il compilatore verifica che non è un tipo e da errore.








\newpage 

