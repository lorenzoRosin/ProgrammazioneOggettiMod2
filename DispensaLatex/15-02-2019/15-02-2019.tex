

\newpage
\section{15-02-2019}
\par

\begin{lstlisting}[basicstyle=\small,]

public interface Iterator<T>{}

\end{lstlisting}

L'interfaccia è un contratto, nel senso che ti promette di fare qualcosa. \newline
In java si scrive il codice ancora prima di sapere cosa andrà a fare. \newline
Il contratto di iteratore è il seguente: \newline
->boolean hasNext(); \newline
->T next(); \newline
->void remove()\newline
Non bisogna sapere necessariamente come sono stati implementati, ti basta sapere che esistono per poter dire che sia un iteratore. \newline
Esempio di definizione di un metodo con iteratore come input: 

\begin{lstlisting}[basicstyle=\small,]

public statics void scorri(Iteratore <Integer> it){
	while(it.hasnext()){
		integer n = it.next();
	}

}

\end{lstlisting}

Esempio di utilizzo 

\begin{lstlisting}[basicstyle=\small,]

scorri(new Iterator<>(){
	...
	...
	...

});

\end{lstlisting}

Quest'ultima è un'espressione, o come meglio dire, un'oggetto fatto al volo. Questa sintassi è stata creata appositamente per le interfacce (dato che non si possono costruire), senza dover andare a definire una classe con la classica implementazione dell'interfaccia. \newline
\textit{ANONYMOUS CLASS} meccanismo comodo per design pattern come le call-back. \newline
Questa implementazione garantisce che la funzione sia \textit{SOUND}, e non crasherà mai a \textit{RunTime}
\newline
IMPLEMENTARE INTERFACCE \newline
1) Con implements: \newline
-> controlla i metodi che hai implementato all'interno della classe \newline
-> assicura che siano implementati tutti \newline
Tipi delle interfacce \newline
Iterator $\Rightarrow$ non è un tipo \newline
Iterator<T> $\Rightarrow$ è un tipo \newline
\textbf{NOTAZIONE BNS} \newline
BNS è il nome della notazione e serve per poter dare delle regole grammaticali. E' una notazione che definisce la sintassi delle espressioni\newline
Iterator da solo, sintatticamente, sarebbe un tipo. Ma il compilatore verifica che non è un tipo e da errore.








\newpage 

