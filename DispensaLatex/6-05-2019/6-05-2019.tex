

\newpage
\section{6-05-2019: MULTI PRODUCER AND MULTI CONSUMER}
\noindent \textbf{APPUNTI DELLA LEZIONE} \newline
Solitamente quando il main thread esce/ritorna, anche tutti gli altri thread da esso generati vengono terminati. In java invece questo non succede: il main thread può terminare ma la Java Virtual Machine termina solo quando sono terminati anche gli altri thread generati dal Main.\newline
Il senso di fare binding (cioè salvare degli oggetti in una variabile) è quello di chiamare più metodi. \newline
Per fare in modo che il main non termini prima degli altri thread da esso generato si può usare il \textit{join}, che invocato su un'oggetto ritorna quando esso viene terminato.
\begin{lstlisting}
Thread t1 = new Thread(new EventThread("e1"));
t1.start();
try {
	  /* ritorna con quando t1 termina */
      t1.join();
} catch (InterruptedException e) {
      e.printStackTrace();
}
\end{lstlisting}
\noindent Come wait e notify, bisogna mettere dentro ad un try e catch anche join. \newline
Usare operazioni di stampa come print in java è un'operazione critica e lenta, perchè vengono richiamati parecchi passaggi. Quindi è meglio non usare mai operazioni di stampa in sezioni critiche.


\noindent \textbf{CODICE FATTO IN CLASSE} \newline
Riporto il codice fatto in classe dal professore. \newline
Penso che in questa lezione abbia spiegato come modificare la nostra classe per fare in modo che supporti un multi consumer-multi producer.

\begin{lstlisting}
/* Classe: Main.java */
package patterns.consumer_producer;

import java.util.ArrayList;
import java.util.List;
import java.util.Random;

public class Main {

    /* private static BlockingQueue<String> q = new LinkedBlockingQueue<>();
     */
    private static List<String> q = new ArrayList<>();
    private static Random rand = new Random();

    public static void print(String s) {
        Thread t = Thread.currentThread();
        System.out.println(String.format("[%s:%d] %s", t.getName(), t.getId(), s));
    }

    public static class Consumer extends Thread {
        @Override
        public void run() {
            while (true) {
                try {
                    Thread.sleep(rand.nextInt(500));
                } catch (InterruptedException e) {
                    e.printStackTrace();
                }

                String s;
                int size;
                synchronized (q) {
                    if (q.isEmpty()) {
                        try {
                            q.wait();
                        } catch (InterruptedException e) {
                            e.printStackTrace();
                        }
                    }
                    s = q.remove(0);
                    size = q.size();
                }
                print(String.format("%s (size = %d)", s, size));
            }
        }
    }

    public static class Producer extends Thread {
        @Override
        public void run() {
            while (true) {
                try {
                    Thread.sleep(rand.nextInt(500));

                    int len = rand.nextInt(10);
                    StringBuilder s = new StringBuilder();
                    for (int i = 0; i < len; ++i) {
                        /* s.append(String.format("%d", i)); */
                        int n = rand.nextInt(26);
                        int m = Character.getNumericValue('a') + n + 87;
                        char c = Character.toChars(m)[0];
                        s.append(c);

                    }
                    print(s.toString());
                    synchronized (q) {
                        q.add(s.toString());
                        q.notify();
                    }

                } catch (Exception e) {
                    e.printStackTrace();
                }
            }
        }
    }

    public static void main(String[] args) {
        for (int i = 0; i < 10; ++i) {
            Producer p = new Producer();
            p.setName(String.format("producer-%d", i));
            p.start();

            Consumer c = new Consumer();
            c.setName(String.format("consumer-%d", i));
            c.start();
        }


        print("byebye");
    }

}
\end{lstlisting}


