

\newpage
\section{28-03-2019: MAP}
\noindent Affinchè due classi siano confrontabili devono implementare il metodo compare to \newline
Solitamente si usa il modificatore protected invece che private in modo tale che da fuori non possano essere modificate, ma possano essere viste quando vengono ereditate. \newline

\noindent \textbf{MAPPE}\newline
Una mappa è un oggetto che mappa chiavi con valori. Non ci possono essere chiavi duplicate e ogni chiave può mappare al più un valore. \newline
Una mappa rappresenta una collection associativa. \newline
Una mappa è parametrica su due tipi di argomenti. Uno rappresenta il dominio (chiave), e l'altro rappresenta il codominio (valore). La gerarchia delle mappe è slegata dalla gerarchia delle collection. \newline
Per la documentazione java sulle mappe consultare questo \href{https://docs.oracle.com/javase/8/docs/api/java/util/Map.html }{SITO} \newline
Una mappa è una collection associativa, con coppie associate a valori.

\noindent Il foreach di JAVA funziona solo con le librerie del JDK originale di JAVA.\newline
Super può essere usato per il costruttore o per metodi che eredito, non è di perse un metodo e non ha un tipo.

\noindent Durante la lezione il professore ha aggiunto nel repository di github il seguente codice: 


\begin{lstlisting}[basicstyle=\small,]
/* Classe: Pair.java */
public class Pair<A, B> {
    private A a;
    private B b;

    public Pair(A a, B b) {
        this.a = a;
        this.b = b;
    }

    public A getFirst() { return a; }
    public B getSecond() { return b; }

}
\end{lstlisting}

\begin{lstlisting}[basicstyle=\small,]
/* Classe: NotFoundException.java */
public class NotFoundException extends Exception {
}
\end{lstlisting}

\begin{lstlisting}[basicstyle=\small,]
/* Classe: MyMap.java */
public interface MyMap<K, V> extends MyCollection<Pair<K, V>> {
	/*K = sono le chiavi */
	/*V = sono i valori  */
    void put(K key, V value);
    V get(K key) throws NotFoundException;

}
\end{lstlisting}

\noindent Decidiamo di implementare la nostra mappa in due maniere diverse! 

\begin{lstlisting}[basicstyle=\small,]
/* Classe: MyListMap__old.java */
import java.util.function.Function;

public class MyListMap__old<K, V> implements MyMap<K, V> {

    private MyList<Pair<K, V>> l;

    public MyListMap__old() {
        this.l = new MyArrayList<>();
    }

    @Override
    public void put(K key, V value) {
        l.add(new Pair<K, V>(key, value));
    }

    @Override
    public V get(K key) throws NotFoundException {
        MyIterator<Pair<K, V>> it = l.iterator();
        while (it.hasNext()) {
            Pair<K, V> p = it.next();
            if (key.equals(p.getFirst())) return p.getSecond();
        }
        throw new NotFoundException();
    }

    @Override
    public void add(Pair<K, V> x) {
        put(x.getFirst(), x.getSecond());
    }

    @Override
    public void clear() {
        l.clear();
    }

    @Override
    public void remove(Pair<K, V> x) {
        l.remove(x);
    }

    @Override
    public boolean contains(Pair<K, V> x) {
        return l.contains(x);
    }

    @Override
    public boolean contains(Function<Pair<K, V>, Boolean> p) {
        return l.contains(p);
    }

    @Override
    public int size() {
        return l.size();
    }

    @Override
    public MyIterator<Pair<K, V>> iterator() {
        return l.iterator();
    }

    @Override
    public int find(Pair<K, V> x) throws Exception {
        return l.find(x);
    }
}
\end{lstlisting}

\begin{lstlisting}[basicstyle=\small,]
/* Classe: MyListMap.java */
public class MyListMap<K, V> extends MyArrayList<Pair<K, V>>
        implements MyMap<K, V> {

    @Override
    public void put(K key, V value) {
        add(new Pair<K, V>(key, value));
    }

    @Override
    public V get(K key) throws NotFoundException {
        MyIterator<Pair<K, V>> it = iterator();
        while (it.hasNext()) {
            Pair<K, V> p = it.next();
            if (key.equals(p.getFirst())) return p.getSecond();
        }
        throw new NotFoundException();
    }
}
\end{lstlisting}

\begin{lstlisting}[basicstyle=\small,]
/* Classe: MyArrayListSortedSet.java */
/* è stata modificata la classe dell'altra volta*/
import java.util.*;
import java.util.function.Function;

public class MyArrayListSortedSet<T extends Comparable<T>>
        extends MyAbstractArrayListSet<T>
        implements MySortedSet<T> {

    public MyArrayListSortedSet() {
        super();
    }

    @Override
    protected void sort() {
        Collections.sort(a);
    }

}
\end{lstlisting}

\begin{lstlisting}[basicstyle=\small,]
/* Classe: MyArrayList.java */
/* è stata modificata la classe dell'altra volta*/
import java.util.function.Function;

public class MyArrayList<T> implements MyList<T> {

    private Object[] a;
    private int actualSize;

    public static class MyException extends Exception {
        public MyException(String s) {
            super(s);
        }
    }

/*    T[] toArray() {
        return (T[]) a;
    }*/

/*    public static Exception returnNow() {
        return new Exception("msg");
    }

    public static void throwNow() throws Exception {
        throw new Exception("msg");
    }

    public static void caller() throws Exception {
        Exception e = returnNow();
        throwNow();
    }

    public static void caller2() {
        try {
            caller();
        }
        catch (Exception e2) {
            // fai qualcosa con e2
        }

    }

    public void m(int x) throws Exception {
        MyException e = new MyException("error messagge");
        if (x < 0) throw e;
    }
  */

    public MyArrayList() {
        clear();
    }

    @Override
    public int find(T x) throws NotFoundException {
        int cnt = 0;
        MyIterator<T> it = iterator();
        while (it.hasNext())
        {
            T y = it.next();
            if (x.equals(y)) return cnt;
            ++cnt;
        }
        throw new NotFoundException();
    }




    public static void main3() {
        MyArrayList<Integer> c = new MyArrayList<>();
        try {
            int index = c.find(6);
            System.out.println("found at index = " + index);
        } catch (NotFoundException e) {
            try {
                int index = c.find(7);
            } catch (NotFoundException e1) {

            }

        }
    }

    @Override
    public boolean contains(T x) {
        for (int i = 0; i < actualSize; ++i) {
            Object o = a[i];
            if (o.equals(x)) return true;
        }
        return false;
    }


    @Override
    public boolean contains(Function<T, Boolean> p) {
        return false;
    }

    @Override
    public int size() {
        return actualSize;
    }


    @Override
    public void clear() {
        a = new Object[100];
        actualSize = 0;
    }

    @Override
    public void add(T o) {
        a[actualSize++] = o;
        if (actualSize >= a.length) {
            Object[] u = new Object[a.length * 2];
            for (int j = 0; j < a.length; ++j)
                u[j] = a[j];
            a = u;
        }
    }

    @Override
    public MyIterator<T> iterator() {
        return new MyIterator<T>() {
            private int pos = 0;

            @Override
            public boolean hasNext() {
                return pos <= actualSize;
            }

            @Override
            public T next() {
                return (T) MyArrayList.this.a[pos++];
            }
        };
    }

    @Override
    public void add(int i, T x) {

    }

    @Override
    public T get(int i) {
        return (T) a[i];
    }

    @Override
    public void set(int i, T x) {
        a[i] = x;
    }

    @Override
    public void remove(T x) {

    }

}

\end{lstlisting}

\begin{lstlisting}[basicstyle=\small,]
/* Classe: Main.java */
/* è stata modificata la classe dell'altra volta*/
import java.util.*;

public class Main {

    public static class Plant {
        private int height;

    }

    public static class Animal implements Comparable<Animal> {
        private int weight;
        private String name;

        public Animal(String name, int w) {
            this.name = name;
            this.weight = w;
        }

        public int getWeight() { return weight; }

        public String getName() {
            return name;
        }

        @Override
        public int compareTo(Animal o) {
            return this.weight - o.weight;
            /*if (this.weight == o.weight) return 0;
            else if (this.weight > o.weight) return 1;
            else return -1;*/
        }
    }

    public static class Dog extends Animal {

        public Dog(String name, int w) {
            super(name, w);
        }
    }


    public static void main2() {
        MyCollection<Pair<String, Integer>> rubrica = new MyListMap<>();
        rubrica.add(new Pair<>("Alvise", 34712345));
        rubrica.add(new Pair<>("Diego", 987654321));
    }


    public static void main(String[] args) {

        List<Animal> a = new ArrayList<>();
        Collections.sort(a);

        List<Plant> b = new ArrayList<>();
        Collections.sort(b, new Comparator<Plant>() {
            @Override
            public int compare(Plant x, Plant y) {
                return x.height - y.height;
            }
        });

    }
}
\end{lstlisting}

\noindent Finita la lezione il professore ha deprecato la classe: MyListMap\_ \_ old.java