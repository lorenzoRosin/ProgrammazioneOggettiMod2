

\newpage
\section{9-05-2019: WRAPPING}
\noindent \textbf{APPUNTI DELLA LEZIONE} \newline
Qunado si ha a che fare con la sincronizzazione è meglio lasciare risolvere questi problemi alla struttura dati che scegliamo di usare, come quando si sceglie di usare una \textit{LinkedBlockingQueue}, dove la sincronizzazione avviene al suo interno. \newline
Il problema dei generics si può risolvere da dentro, mettendo un vincolo sui generics dicendo che devono implementare comparable. Oppure da fuori, fornendo una primitiva che permetta il confronto: comparator. \newline
\begin{lstlisting}
/* eccone un esempio da dentro */
static <T extends Comparable<T>> void quickSort(List<T>l)
/* eccone un esempio da fuori */
static <T> void quickSort(List<T>l,  BiFunction<T, T, Integer> f){
	T a = l.get(0), b = l.get(1);
	if(f.apply(a,b)<0)
	.......
}
\end{lstlisting}

\noindent Esempi di wrapping
\begin{lstlisting}
static <T> void quickSort(List<T> l, Comparator<T> f){
	T a = l.get(0), b = l.get(1);
	if(f.compare(a,b)<0)
	.......
}

static <T> void quickSort2(List<T> l, BiFunction<T, T, Integer> f){
	quickSort(l, new Comparator<T>(){
		public int Compare(T o1, T, o2){
			return f.apply(01,02);
		}
	});
}

static <T> void quickSort(List<T> l, BiFunction<T, T, Boolean> f){
	quickSort(l, new Comparator<T>(){
		public int Compare(T o1, T, o2){
			return f.apply(01,02)? -1 : 1;
		}
	});
}

\end{lstlisting}


\noindent \textbf{CODICE FATTO IN CLASSE} \newline
\noindent Ha modificato ulteriormente la classe dell'altra volta aggiungendo del codice in testa. Riporto solo il codice nuovo.
\begin{lstlisting}
package patterns.consumer_producer;

import java.util.ArrayList;
import java.util.Comparator;
import java.util.List;
import java.util.Random;
import java.util.function.BiFunction;
import java.util.function.Function;

public class Main {

    int f(int x) {
        return this.f("ciao");
    }

    int f(String x) {
        return this.f(1);
    }

    static <T> void quicksort3(List<? extends Comparable<T>> l) {

    static <T extends Comparable<T>> void quicksort3(List<T> l) {
        T a = l.get(0), b = l.get(1);
        if (a.compareTo(b) < 0)

    }

    static void quicksort3(List<? extends Comparable<?>> l) {
        Object a = l.get(0), b = l.get(1);
        if (a. < 0)

    }

    static void quicksort(List<Comparable<Comparable<C>>> l) {
    }

    static <T> void quicksort(List<T> l, Comparator<T> f) {
        T a = l.get(0), b = l.get(1);
        if (f.compare(a, b) < 0)
        /* algo */
    }

    static <T> void quicksort2(List<T> l, BiFunction<T, T, Integer> f) {
        quicksort(l, new Comparator<T>() {
            @Override
            public int compare(T o1, T o2) {
                return f.apply(o1, o2);
            }
        });

    }

    static <T> void quicksort(List<T> l, BiFunction<T, T, Boolean> f) {
        quicksort(l, new Comparator<T>() {
            @Override
            public int compare(T o1, T o2) {
                return f.apply(o1, o2) ? -1 : 1;
            }
        });
    }

    /* private static BlockingQueue<String> q = new LinkedBlockingQueue<>();
     */
    private static List<String> q = new ArrayList<>();......
    ......    
\end{lstlisting}