\newpage
\section{16-05-2019: ULTIMA LEZIONE}
\noindent Mancano gli appunti dell'ultima mezzora, ma ormai penso sia impossibile recuperarli. \newline

\noindent \textbf{SUB-TYPING} \newline
\textbullet\ \textit{Subtyping}: polimorfismo verticale, si eredita. I tipi si allargano. \newline
\textbullet\ \textit{Generics}: polimorfismo orizzontale \newline
\noindent Esempio:
\begin{lstlisting}
public static object ident(Object o){
	return o;
}
/* posso passare qualsiasi cosa ma.. */
String s = "ciao";

/* Questomi da errore in quanto ident restituisce Object */
String s2 = ident(s); 

/* posso risolvere con un down cast */
String s3 = (String) ident(s);

/* Andare su con i tipi è gratuito, scendere no*/
 public static <T> T ident(T o){
 	return o;
 }
/* T è un tipo generale e può essere diverso ad ogni
 * invocazione */
\end{lstlisting}


\noindent Ogni volta che si chiama un metodo con i generics questo viene sostituito con il vero tipo, in tutta la firma del metodo.
\begin{lstlisting}
/* in questo caso T sarà sostituito con String*/
String s4 = ident(s);


/* Esempio: posso passare String ma anche sottotipi di
 * String  */
public static <T extends String> T ident (T o)
\end{lstlisting}

\noindent \textbf{COLLECTION} \newline
Le interfacce esprimono contratti ma tramite il type system non si può definire cosa poi farà il metodo (anche se ne definisco il comportamento).\newline
Un'interfaccia è una classe astratta in cui tutti i metodi sono non implementati. \newline
un metodo di default è un'implementazione di metodo nell'interfaccia. \newline
Quando si parla di ereditarietà multipla si intende l'ereditarietà tra le classi (PER ESAME) \newline
Eccezioni: oggetti che si possono lanciare tramite il costrutto throw, servono a rappresentare stati d'errore.