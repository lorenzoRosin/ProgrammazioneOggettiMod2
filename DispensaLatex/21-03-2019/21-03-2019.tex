

\newpage
\section{21-03-2019}
\textbullet\ -> ITERABLE:Posso solo scorrere \newline
\textbullet\ ---> COLLECTION: Posso scorrere, aggiungere e togliere \newline
\textbullet\ -----> LIST: Posso scorrere e aggiungere o togliere con un indice \newline
\textbullet\ -------> ARRAYLIST \newline
ArrayList è sottotipo di list, in quanto è una classe che implementa list! \newline
\digraph{ao}{rankdir=LR;

   a [label="Iterable" shape = "record"]; 
   b [label="Collection" shape = "record"]; 
   c [label="Set" shape = "record"]; 
   d [label="Hash" shape = "record"]; 
   e [label="List" shape = "record"]; 
   f [label="ArrayList" shape = "record"];    
   b->a; e->b; f->e; d->c; c->b} \newline

\noindent Caratteristiche dell'interfaccia set:\newline
\textbullet\ Gli elementi non sono duplicati \newline
\textbullet\ Gli elementi non sono ordinati in base a come sono inseriti \newline	
\textbullet\ Non vengono inseriti metodi nuovi, eredita solo quelli del padre \newline
\textbullet\ I metodi nuovi vengono messi nella classe che implementa l'interfaccia \newline

\noindent Per evitare di riprodurre codice si usano le classi astratte dalle quali poi si erediterà. 

\noindent Relazione di ordinamento: operatore binario che permette di mappare elementi di due insiemi diversi.

\noindent Una classe con metodi tutti statici non si può costruire. Rappresenta dunque un contenitore di metodi (è un pezzo di libreria).

\noindent Il professore questo giorno ha caricato su github del codice chiamato: TinyJDK. Lo riporto per completezza:
\begin{lstlisting}[basicstyle=\small,]
/* Classe: MyIterable.java */
public interface MyIterable<E> {

    MyIterator<E> iterator();
    int find(E x) throws Exception;

}
\end{lstlisting}

\begin{lstlisting}[basicstyle=\small,]
/* Classe: MyIterator.java */
public interface MyIterator<E> {
    boolean hasNext();
    E next();
}
\end{lstlisting}

\begin{lstlisting}[basicstyle=\small,]
/* Classe: MyCollection.java */
import java.util.Collection;
import java.util.function.Function;

public interface MyCollection<T> extends MyIterable<T> {
    void add(T x);
    void clear();
    void remove(T x);   // da decidere se ci piace o no
    boolean contains(T x);
    boolean contains(Function<T, Boolean> p);
    int size();


}
\end{lstlisting}

\begin{lstlisting}[basicstyle=\small,]
/* Classe: MyList.java */

public interface MyList<T> extends MyCollection<T> {
    void add(int i, T x);
    T get(int i);
    void set(int i, T x);
}
\end{lstlisting}

\begin{lstlisting}[basicstyle=\small,]
/* Classe: MySet.java */
public interface MySet<T> extends MyCollection<T> {
}
\end{lstlisting}

\begin{lstlisting}[basicstyle=\small,]
/* Classe: MyArrayList.java */
import java.util.Collection;
import java.util.function.Function;

public class MyArrayList<T> implements MyList<T> {

    private Object[] a;
    private int actualSize;

    public static class MyException extends Exception {
        public MyException(String s) {
            super(s);
        }
    }

/*    T[] toArray() {
        return (T[]) a;
    }*/

/*    public static Exception returnNow() {
        return new Exception("msg");
    }

    public static void throwNow() throws Exception {
        throw new Exception("msg");
    }

    public static void caller() throws Exception {
        Exception e = returnNow();
        throwNow();
    }

    public static void caller2() {
        try {
            caller();
        }
        catch (Exception e2) {
            // fai qualcosa con e2
        }

    }

    public void m(int x) throws Exception {
        MyException e = new MyException("error messagge");
        if (x < 0) throw e;
    }
  */

    public MyArrayList() {
        clear();
    }

    public static class NotFoundException extends Exception {
    }

    @Override
    public int find(T x) throws NotFoundException {
        int cnt = 0;
        MyIterator<T> it = iterator();
        while (it.hasNext())
        {
            T y = it.next();
            if (x.equals(y)) return cnt;
            ++cnt;
        }
        throw new NotFoundException();
    }




    public static void main3() {
        MyArrayList<Integer> c = new MyArrayList<>();
        try {
            int index = c.find(6);
            System.out.println("found at index = " + index);
        } catch (NotFoundException e) {
            try {
                int index = c.find(7);
            } catch (NotFoundException e1) {

            }

        }
    }

    @Override
    public boolean contains(T x) {
        for (int i = 0; i < actualSize; ++i) {
            Object o = a[i];
            if (o.equals(x)) return true;
        }
        return false;
    }


    @Override
    public boolean contains(Function<T, Boolean> p) {
        return false;
    }

    @Override
    public int size() {
        return actualSize;
    }


    @Override
    public void clear() {
        a = new Object[100];
        actualSize = 0;
    }

    @Override
    public void add(T o) {
        a[actualSize++] = o;
        if (actualSize >= a.length) {
            Object[] u = new Object[a.length * 2];
            for (int j = 0; j < a.length; ++j)
                u[j] = a[j];
            a = u;
        }
    }

    @Override
    public MyIterator<T> iterator() {
        return new MyIterator<T>() {
            private int pos = 0;

            @Override
            public boolean hasNext() {
                return pos <= actualSize;
            }

            @Override
            public T next() {
                return (T) MyArrayList.this.a[pos++];
            }
        };
    }

    @Override
    public void add(int i, T x) {

    }

    @Override
    public T get(int i) {
        return (T) a[i];
    }

    @Override
    public void set(int i, T x) {
        a[i] = x;
    }

    @Override
    public void remove(T x) {

    }

}
\end{lstlisting}

\begin{lstlisting}[basicstyle=\small,]
/* Classe: MyArrayListSet.java */
import java.util.ArrayList;
import java.util.Arrays;
import java.util.Collections;
import java.util.Comparator;
import java.util.function.Function;

public class MyArrayListSet<T extends Comparable<T>> implements MySet<T> {
    private final Comparator<T> p;
    private final ArrayList<T> a;

    public MyArrayListSet(Comparator<T> p) {
        this.a = new ArrayList<T>();
        this.p = p;
    }

    @Override
    public void add(T x) {
        if (!contains(x)) {
            a.add(x);
            sort();
        }
    }

    private void sort() {
        Collections.sort(a, p);
    }

    @Override
    public void clear() {
        a.clear();
    }

    @Override
    public void remove(T x) {
        a.remove(x);
    }

    @Override
    public boolean contains(T x) {
        return a.contains(x);
    }

    @Override
    public boolean contains(Function<T, Boolean> p) {
        return a.contains(p);
    }

    @Override
    public int size() {
        return a.size();
    }

    @Override
    public MyIterator<T> iterator() {
        return a.iterator();
    }

    @Override
    public int find(T x) throws Exception {
        return a.find(x);
    }
}
\end{lstlisting}





