

\newpage
\section{8-04-2019}
\noindent Non ho ancora messo apposto gli appunti di quel giorno però intanto importo il codice scritto dal professore: 

THREAD

\begin{lstlisting}[basicstyle=\small,]
/* Classe: Main.java */
package threads;

public class Main {

    private static void loop(int n, int ms) {
        for (int i = 0; i < n; ++i) {
            System.out.println(String.format("thread[%d]: #%d", Thread.currentThread().getId(), i));
            try {
                Thread.sleep(ms);
            } catch (InterruptedException e) {
                e.printStackTrace();
            }
        }
    }

    public static class MyThread extends Thread {
        private final int n;

        public MyThread(int n) {
            this.n = n;
        }

        @Override
        public void run() {
            Main.loop(n, 300);
        }

    }


    public static void main(String[] args) {
        MyThread th = new MyThread(23);
        th.start();
        th.run();
        loop(11, 500);
    }

}


\end{lstlisting}

 
SINGLETON
 
\begin{lstlisting}[basicstyle=\small,]
/* Classe: Main.java */
package patterns.singleton;

public class Main {

    public static void main(String[] args) {
        Singleton single1 = Singleton.getInstance();
        Singleton single2 = Singleton.getInstance();
        System.out.println("is it the same object? " + (single1 == single2));
    }

}

\end{lstlisting}

\begin{lstlisting}[basicstyle=\small,]
/* Classe: Singleton.java */
package patterns.singleton;

class Singleton {
    private static Singleton instance = null;

    // costruttore privato per non permettere a nessuno di costruire questa classe
    private Singleton() {}

    // metodo statico che funge da pseudo-costruttore
    public static Singleton getInstance() {
        if (instance == null)
            instance = new Singleton();
        return instance;
    }
}

\end{lstlisting}