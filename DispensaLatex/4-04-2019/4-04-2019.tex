

\newpage
\section{4-04-2019: SINGLETON E THREAD}
\textbf{DESIGN PATTERN SINGLETON} \newline
Si limita l'istanzazione di una classe a uan sola istanza. \newline
Può servire per creare oggetti statici (per esempio un campo statico esiste anche prima dell'istanza dell'oggetto del tipo della classe a cui appartiene). \newline
Un costruttore privato può essere accessibile solo dalla classe, e non dall'esterno. \newline
I metodi statici che ritornano lo stesso tipo del costruttore della classe sono costruttori travestiti. \newline
i costruttori sono metodi d'istanza. Non serve il modificatore di accesso static, anche se sono simili a metodi statici. \newline
I metodi statici sono pseudo costruttori, se usati nel modo opportuno. \newline
Un metodo statico è un metodo che non ha bisogno di istanza. un campo statico si può vedere anche se non ha istanza (da un metodo statico si può accedere solo a campi statici, poichè non ha riferimento al this). \newline
In questo corso non andremo oltre con questo design pattner perchè sempre di più in programmazione si sta andando verso la programmazione funzionale (in contrapposizione a quella imperattiva).


\begin{lstlisting}[basicstyle=\small,]
/* Classe: Main.java */
package patterns.singleton;

public class Main {

    public static void main(String[] args) {
        Singleton single1 = Singleton.getInstance();
        Singleton single2 = Singleton.getInstance();
        System.out.println("is it the same object? " + (single1 == single2));
    }

}

\end{lstlisting}

\begin{lstlisting}[basicstyle=\small,]
/* Classe: Singleton.java */
package patterns.singleton;

class Singleton {
    private static Singleton instance = null;

    // costruttore privato per non permettere a nessuno di costruire questa classe
    private Singleton() {}

    // metodo statico che funge da pseudo-costruttore
    public static Singleton getInstance() {
        if (instance == null)
            instance = new Singleton();
        return instance;
    }
}

\end{lstlisting}


\noindent \textbf{INTERFACCIA GRAFICA} \newline
Un widget è un supertipo dell'interfaccia grafica. \newline
Busy-Loop: ciclo che cicla a vuoto, non restituendo il controllo al S.O. \newline
Le interfacce grafiche non si programmano a busy-loop. Si chiede al sistema operativo di "svegliarmi quando accade un evento". Dunque si programmano a suon di call-back. \newline
un esempio di call-back è il ctr-c per interrompere un programma in una shell linux. Questa combinazione di tasti manda un segnale, ma in realtà viene invocata una call-back. \newline
programmazion ad eventi: si programma con la call-back che vengono chaimate quando opportuno.

\noindent \textbf{THREAD IN JAVA} \newline
Esiste una classe Thread che si eredita e si ovverrida.



\begin{lstlisting}[basicstyle=\small,]
/* Classe: Main.java */
package threads;

public class Main {

    private static void loop(int n, int ms) {
        for (int i = 0; i < n; ++i) {
            System.out.println(String.format("thread[%d]: #%d", Thread.currentThread().getId(), i));
            try {
                Thread.sleep(ms);
            } catch (InterruptedException e) {
                e.printStackTrace();
            }
        }
    }

    public static class MyThread extends Thread {
        private final int n;

        public MyThread(int n) {
            this.n = n;
        }

        @Override
        public void run() {
            Main.loop(n, 300);
        }

    }


    public static void main(String[] args) {
        MyThread th = new MyThread(23);
        th.start();
        th.run();
        loop(11, 500);
    }

}


\end{lstlisting}

 

 
