

\newpage
\section{2-05-2019: STRING BUILDER}
\noindent \textbf{ESAME 4-09-2018: SINGLETON} \newline
Riporto il codice fatto in classe dal professore, che risolve il primo esercizio dell'esame del 4-09-2018.
\begin{lstlisting}
/* Classe: RandomSingleton.java */

package patterns.singleton;

import org.jetbrains.annotations.NotNull;
import org.jetbrains.annotations.Nullable;

import java.util.Random;

class RandomSingleton {
    @Nullable
    private static RandomSingleton instance = null;
    @NotNull
    private Random rand;

    /* i costruttori sono PRIVATI, per non
     * permettere a nessuno di costruire
     * questa classe senza passare per i 
     * metodi statici getInstance() */
    private RandomSingleton() {
        rand = new Random();
    }

    /* altro costruttore privato con parametro */
    private RandomSingleton(int seed) {
        rand = new Random(seed);
    }

    /* metodo statico che funge da pseudo-costruttore */
    @NotNull
    public static RandomSingleton getInstance() {
        if (instance == null)
            instance = new RandomSingleton();
        return instance;
    }

    /* altro metodo statico per avere l'istanza, 
     * ma questa volta con un parametro */
    @NotNull
    public static RandomSingleton getInstance(int seed) {
        if (instance == null)
            instance = new RandomSingleton(seed);   
            /* qui costruiamo l'oggetto con il costruttore
             * che prende il seed come parametro */
        return instance;
    }

}
\end{lstlisting}


\begin{lstlisting}
/* Classe: Main.java */
package patterns.singleton;

public class Main {

    public static void main(String[] args) {
        RandomSingleton o1 = RandomSingleton.getInstance();
        RandomSingleton o2 = RandomSingleton.getInstance();
        System.out.println("are these the same object? " + (o1 == o2 ? "yes" : "no"));
    }

}
\end{lstlisting}

\noindent \textbf{STRING BUILDER} \newline
Anche lo standard output usa dei mutex per evitare che vari thread si sovrascrivano del tutto quando usano delle funzioni come \textit{println}. \newline
System è una classe, con all'interno dei campi statici. Out è un oggetto messo dentro un oggetto statico della classe System.out, che ha tipo Printstream. \newline
StrinBuilder è una classe già esistente di java che permette di appendere in una coda stringhe in modo efficiente, senza doverne generare di nuove ogni volta. Infatti la semplice operazione: s = s + "ciao" va a creare varie stringhe! 


\noindent \textbf{CODICE FATTO IN CLASSE: CONSUMER-PRODUCER} \newline
Riporto il codice fatto in classe dal professore. \newline
Il professore ha rimosso la classe ThreadPool che era stata messa nella cartella WORK. Ora esiste solo il file ThreadPool all'interno del TinySDK. \newline


\begin{lstlisting}
/* Classe: Main.java */
/* Ha modificato la classe dell'altra
 * volta, aggiungendo la gestione dello
 * StringBuilder */

package patterns.consumer_producer;

import java.io.PrintStream;
import java.util.Random;
import java.util.concurrent.BlockingQueue;
import java.util.concurrent.LinkedBlockingQueue;

public class Main {

    private static BlockingQueue<String> q = new LinkedBlockingQueue<>();
    private static Random rand = new Random();

    public static void print(String s) {
        Thread t = Thread.currentThread();
        System.out.println(String.format("[%s:%d] %s", t.getName(), t.getId(), s));
    }

    public static class Consumer extends Thread {
        @Override
        public void run() {
            while (true) {
                try {
                    Thread.sleep(rand.nextInt(500));

                    String s = q.take();
                    print(s);

                } catch (InterruptedException e) {
                    e.printStackTrace();
                }
            }
        }
    }

    public static class Producer extends Thread {
        @Override
        public void run() {
            while (true) {
                try {
                    Thread.sleep(rand.nextInt(500));

                    int len = rand.nextInt(10);
                    StringBuilder s = new StringBuilder();
                    for (int i = 0; i < len; ++i) {
                        s.append(String.format("%d", i));
						/* int n = rand.nextInt(26);
                        * int m = Character.getNumericValue('a') + n;
                        * char c = Character.toChars(m)[0];
                        * s = s + c; */
                    }
                    print(s.toString());
                    q.add(s.toString());

                } catch (Exception e) {
                    e.printStackTrace();
                }
            }
        }
    }

    public static void main(String[] args) {
            new Producer().start();
            new Consumer().start();
            print("byebye");
    }

}

\end{lstlisting}

















